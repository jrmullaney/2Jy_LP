%%%%%%%%%%%%%%%%%%%%%%%%%%%%%%%%%%%%%%%%%%%%%%%%%%%%%%%%%%%%%%%%%%%%%%
%
%.IDENTIFICATION $Id: templatelarge.tex.src,v 1.29 2008/01/25 10:47:12 fsogni Exp $
%.LANGUAGE       TeX, LaTeX
%.ENVIRONMENT    ESOFORM
%.PURPOSE        Template application form for ESO Observing time.
%.AUTHOR         The Esoform Package is maintained by the Observing
%                Programmes Office (OPO) while the background software
%                is provided by the User Support System (USS) Department.
%
%-----------------------------------------------------------------------
%
%
%                   ESO LA SILLA PARANAL OBSERVATORY
%                   --------------------------------
%                   LARGE PROGRAMME PHASE 1 TEMPLATE
%                   --------------------------------
%
%
%
%          PLEASE CHECK THE ESOFORM USERS' MANUAL FOR DETAILED 
%              INFORMATION AND DESCRIPTIONS OF THE MACROS. 
%     (see the file usersmanual.tex provided in the ESOFORM package) 
%
%
%        ====>>>> TO BE SUBMITTED THROUGH WEB UPLOAD  <<<<====
%               (see the Call for Proposals for details)
%
%%%%%%%%%%%%%%%%%%%%%%%%%%%%%%%%%%%%%%%%%%%%%%%%%%%%%%%%%%%%%%%%%%%%%%

%%%%%%%%%%%%%%%%%%%%%%%%%%%%%%%%%%%%%%%%%%%%%%%%%%%%%%%%%%%%%%%%%%%%%%
%
%                      I M P O R T A N T    N O T E
%                      ----------------------------
%
% By submitting this proposal, the Principal Investigator takes full
% responsibility for the content of the proposal, in particular with
% regard to the names of CoI's and the agreement to act in accordance
% with the ESO policy and regulations, should observing time be
% granted.
%
%%%%%%%%%%%%%%%%%%%%%%%%%%%%%%%%%%%%%%%%%%%%%%%%%%%%%%%%%%%%%%%%%%%%%% 

%
%    - LaTeX *is* sensitive towards upper and lower case letters.
%    - Everything after a `%' character is taken as comments.
%    - DO NOT CHANGE ANY OF THE MACRO NAMES (words beginning with `\')
%    - DO NOT INSERT ANY TEXT OUTSIDE THE PROVIDED MACROS
%

%
%    - All parameters are checked at the verification or submission.
%    - Some parameters are also checked during the pdfLaTeX
%      compilation.  If this is not the case, this is indicated by the
%      phrase
%      "This parameter is NOT checked at the pdfLaTeX compilation."
%

\documentclass{esoformlarge}

% The list of LaTeX definitions of commonly used astronomical symbols
% is already included in the style file common2e.sty (see Table 1 in
% the Users' Manual).  If you have your own macros or definitions,
% please insert them here, between the \documentclass{esoformlarge}
% and the \begin{document} commands.
%
%     PLEASE USE NEITHER YOUR OWN MACROS NOR ANY TEX/LATEX MACROS  
%       IN THE \Title, \Abstract, \PI, \CoI, and \Target MACROS.
%
% WARNING: IT IS THE RESPONSIBILITY OF THE APPLICANTS TO STAY WITHIN THE
% CURRENT BOX LIMITS AND ELIMINATE POTENTIAL OVERFILL/OVERWRITE PROBLEMS 

\usepackage{xcolor}

\begin{document}

%%%%%%%%%%%%%%%%%%%%%%%%%%%%%%%%%%%%%%%%%%%%%%%%%%%%%%%%%%%%%%%%%%%%%%%%
%%%%% CONTENTS OF THE FIRST PAGE %%%%%%%%%%%%%%%%%%%%%%%%%%%%%%%%%%%%%%%
%%%%%%%%%%%%%%%%%%%%%%%%%%%%%%%%%%%%%%%%%%%%%%%%%%%%%%%%%%%%%%%%%%%%%%%%
%
%---- BOX 1 ------------------------------------------------------------
%
% You should use this template for period 100A applications ONLY.
%
% DO NOT EDIT THE MACRO BELOW. 

\Cycle{100A}

% Type below, within the curly braces {}, the title of your observing
% programme (up to two lines).
% This parameter is NOT checked at the pdfLaTeX compilation.
%
% DO NOT USE ANY TEX/LATEX MACROS IN THE TITLE

\Title{RAMS: The Radio-AGN MUSE Survey: Determining the triggering and
  impact of a benchmark sample of powerful radio Active Galactic Nuclei.}  

% Type below the numeric code corresponding to the subcategory of your
% programme.

\SubCategoryCode{A9}   


%This is the only type valid with this template. Do NOT change it.
\ProgrammeType{LARGE}

% For GTO proposals only: uncomment the following and fill out the GTO
% programme code (as communicated to the respective GTO coordinator).

%\GTOcontract{INS-consortium}


%---- BOX 2 ------------------------------------------------------------
%
% Type below a concise abstract of your proposal (up to 13 lines).
% This parameter is NOT checked at the pdfLaTeX compilation.
%
% DO NOT USE ANY TEX/LATEX MACROS IN THE ABSTRACT

\Abstract{

  Active Galactic Nuclei (AGNs) are believed to regulate galaxy growth
  by injecting vast amounts of energy into their host galaxies and
  halos which shuts-off star formation. This ``AGN feedback'' is
  supported by observations of powerful outflows launched from AGNs
  into their immediate surroundings. However, we still do not know how
  AGNs are triggered, nor what drives their outflows. This raises
  major uncertainties regarding whether AGN feedback is accurately
  implemented in models of galaxy evolution. To address this requires
  mapping the kinematics of AGN outflows and host galaxies at high
  spatial resolutions, which can only be achieved via IFU observations
  of nearby AGNs. To date, all IFU surveys of nearby AGNs have
  focussed on radio-weak AGNs, yet there is strong evidence that
  radio-powerful AGNs play a major role in driving feedback-inducing
  outflows. It is likely, therefore, that IFU surveys of AGNs have so
  far ignored one of the most important drivers of AGN feedback. We
  propose to address this by using MUSE to conduct the first IFU
  survey of a representative sample of local, radio-powerful AGN. From
  this, we will determine (a) how powerful radio AGNs are triggered;
  (b) the role of radio jets vs. radiative driving in powering
  feedback-inducing AGN outflows; and (c) the mass/ momentum/energy
  content of these outflows, and whether they have a significant
  impact on their host galaxies.

}

%---- BOX 3 ------------------------------------------------------------
%
% Description of the observing run(s) necessary for the completion of
% your programme.  The macro takes ten parameters: run ID, period,
% instrument, time requested, month preference, moon requirement,
% seeing requirement, transparency requirement, observing mode and 
% run type.
%
% 1. RUN ID
% Valid values: A, B, ..., Z
%
% 2. PERIOD
% For Paranal and APEX instruments four periods are allowed,
% hence the valid range is from 100 to 103.
% La Silla large proposals can request up to Period 97.
% This parameter is NOT checked at the pdfLaTeX compilation.
%
% 3. INSTRUMENT
% Valid values: EFOSC2 FLAMES FORS2 HARPS KMOS LABOCA MUSE OMEGACAM PIONIER SEPIA SHFI SINFONI SOFI SOFOSC SPHERE Special3.6 SpecialNTT UVES XSHOOTER
%
% 4. TIME REQUESTED
% In hours for Service Mode, in nights for Visitor Mode.
% In either case the time can be rounded up to  1 decimal place.    
% This parameter is NOT checked at the pdfLaTeX compilation.
% 
% 5. MONTH PREFERENCE
% Valid values: apr, may, jun,
% jul, aug, sep, oct,
% nov, dec, jan, feb,
% mar, any
%
% 6. MOON REQUIREMENT
% Valid values: d, g, n
%
% 7. SEEING REQUIREMENT
% Valid values: 0.4, 0.6, 0.8, 1.0, 1.2, 1.4, n
%
% 8. TRANSPARENCY REQUIREMENT
% Valid values: CLR, PHO, THN
%
% 9. OBSERVING MODE
% Valid values: v, s
% 
% 10. RUN TYPE 
% Please leave blank for all Large Programme proposals.

\ObservingRun{A}{100}{MUSE}{21.7h}{any}{g}{1.0}{PHO}{s}{}
\ObservingRun{B}{101}{MUSE}{33.1h}{any}{g}{1.0}{PHO}{s}{}
\ObservingRun{C}{102}{MUSE}{21.7h}{any}{g}{1.0}{PHO}{s}{}
\ObservingRun{D}{103}{MUSE}{34.1h}{any}{g}{1.0}{PHO}{s}{}

% Proprietary time requested.
% Valid values: % 0, 1, 2, 6, 12

\ProprietaryTime{12}


%---- BOX 4 ------------------------------------------------------------
% Please provide the ESO User Portal username for the Principal 
% Investigator (PI) in the \PI field.
%
% For the Co-I's (CoI) please fill in the following details:
% First and middle initials, family name, the institute code
% corresponding to their affiliation.
% The corresponding affiliation should be entered for EACH
% Co-I separately in order to ensure the correct details of 
% all Co-I's are stored in the ESO database.
% You can find all institute codes listed according to country
% on the following webpage: 
% http://www.eso.org/sci/observing/phase1/countryselect.html
%
% DO NOT USE ANY TEX/LATEX MACROS HERE
%

\PI{jmullaney} 
% Replace with PI's ESO User Portal username.

\CoI{C.}{Tadhunter}{1245}
\CoI{D.}{Alexander}{1237}
\CoI{E.}{Bernhard}{1245}
\CoI{P.}{Bessiere}{1824}
\CoI{D.}{Dicken}{1193}
\CoI{C.}{Harrison}{1258}
\CoI{R.}{Morganti}{1096}
\CoI{C.}{Ramos-Almeida}{1393}
\CoI{D.}{Rosario}{1237}
\CoI{M.}{Rose}{1245}
\CoI{F.}{Santoro}{1096}
\CoI{R.}{Spence}{1245}
\CoI{M.}{Villar-Martin}{8545}

% Please note:
% Due to the way in which the proposal receiver system parses
% the CoI macro, the number of pairs of curly brackets '{}'
% in this macro MUST be strictly equal to 3, i.e., the
% number of parameters of the macro. Accordingly, curly
% brackets should not be used within the parameters (e.g.,
% to protect LaTeX signs).
%
% For instance:
% \CoI{L.}{Ma\c con}{1098}
% \CoI{R.}{Men\'endez}{1098}
%
% are valid, while
%
% \CoI{L.}{Ma{\c}con}{1098}
% \CoI{R.}{Men{\'}endez}{1098}
%
% are not. Unfortunately the receiver does not give an
% explicit error message when such invalid forms are
% used in the CoI macro, but the processing of the proposal
% keeps hanging indefinitely.


%%%%%%%%%%%%%%%%%%%%%%%%%%%%%%%%%%%%%%%%%%%%%%%%%%%%%%%%%%%%%%%%%%%%%%%%
%%%%% THE THREE PAGES OF THE SCIENTIFIC DESCRIPTION %%%%%%%%%%%%%%%%%%%%
%%%%%%%%%%%%%%%%%%%%%%%%%%%%%%%%%%%%%%%%%%%%%%%%%%%%%%%%%%%%%%%%%%%%%%%%
%
%---- BOX 5 ------------------------------------------------------------
%
%               THIS DESCRIPTION IS RESTRICTED TO THREE PAGES 
%
%   THE RELATIVE LENGTHS OF EACH OF THESE FIVE SECTIONS ARE VARIABLE,
%                BUT THEIR SUM IS RESTRICTED TO THREE PAGES
%
% All macros in this box are NOT checked at the pdfLaTeX compilation.

\ScientificRationale{
  % Scientific rationale: scientific background of the project,
  % pertinent references; previous work plus justification for present
  % proposal.

  {\bf \large The case for AGN feedback}

  Determining how today's galaxies have grown and evolved to their
  present state is the primary goal of extragalactic research. It is
  now clear that galaxy growth is strongly regulated by so-called
  ``feedback'' processes (e.g., Vogelsberger et al. 2014, Schaye et
  al. 2015). Among the most important of these is the suppression of
  galaxy growth by Active Galactic Nuclei (AGNs) that heat and/or
  expel gas which would otherwise collapse to form stars (see Fabian,
  2012; Harrison, 2017 for reviews). Without this ``AGN feedback'',
  models of galaxy growth fail to reproduce the masses, shapes and
  sizes of the observed galaxy population. As such, {\bf it is widely
    considered that AGNs have played a major role in shaping today's
    galaxies}.
  
  \vspace{1.0mm}
  
  Recently, models invoking AGN feedback have received significant
  empirical support from observations. Firstly, X-ray observations of
  nearby clusters have revealed AGNs injecting considerable amounts of
  energy into the intergalactic medium, preventing it from cooling and
  forming stars (e.g., McNamara \& Nulsen 2012). Secondly, there is
  clear evidence of fast (i.e., ${\rm >500s~km~s^{-1}}$), ionised
  outflows in the optical and near-infrared spectra of a significant
  fraction (i.e., $\sim20\%$; e.g., Mullaney et al. 2013; Harrison et
  al. 2014) of all AGNs. Both provide strong evidence of energy
  transport from AGNs, as required by feedback models.  The problem we
  face, however, is that we do still not understand (a) how AGNs are
  triggered and (b) what mechanism drives the outflows that transport
  energy from the AGN. Until these questions are addressed, it will
  remain impossible to test whether AGN feedback is being accurately
  implemented in models of galaxy growth.

  \vspace{1.0mm}

   {\bf \large The role of IFU surveys in studies of AGN feedback}

   The key to determining how AGNs are triggered are spatially
   resolved kinematics of their host galaxies. This is because any
   triggering mechanism must funnel gas to fuel the AGN from galactic
   scales to the nucleus. Similarly, to determine how AGN outflows are
   driven requires spatially resolved observations of those outflows,
   mapped-out by their gas kinematics. Such spatially resolved
   kinematics of the outflows and, in particular, the host galaxy can
   {\it only} be delivered by integral field (IFU) observations of
   {\it low redshift} (i.e., $z<0.4$) AGNs. Further, since AGNs and
   their host galaxies display a lot of diversity, it is important to
   conduct IFU surveys of sufficient numbers of AGNs to identify
   underlying trends with, e.g., luminosity, mass, host morphology
   etc.

  \vspace{1.0mm}

  To date, the only large IFU survey of nearby AGNs has focussed
  radio-weak AGNs (i.e., CARS $[$PI: Husemann$]$). However, {\bf there
    is strong evidence of an association between fast, ionised
    outflows and radio-powerful AGNs}, with 50\% of AGNs with 1.4~GHz
  radio luminosities ($L_{\rm 1.4GHz}$) above
  ${\rm 10^{23}~W~Hz^{-1}}$ displaying evidence of powerful
  (${\rm >500~km~s^{-1}}$) outflows, compared to just 10\% of those
  below this radio luminosity threshold (Mullaney et al. 2013; Fig. 1,
  {\it left}). Furthermore, our own resolved long-slit and IFU
  observations of nearby radio AGNs show clear signs of interaction
  between radio jets and the ISM (e.g., Holt et al. 2008, Tadhunter et
  al. 2014, Santoro et al. 2015). This raises the prospect that jets
  launched from radio AGNs are an important driver of powerful gas
  outflows, as predicted by the hydrodynamical simulations of jet/gas
  interactions described in Wagner et al. (2011; Fig. 1 {\it right};
  also Mukherjee et al. 2016). Furthermore, radio loud AGNs are the
  {\it only} type capable of inducing ``radio-mode'' AGN feedback,
  which is thought to heat intergalactic gas, preventing it from
  collapsing onto galaxies to form stars (e.g., Bower et
  al. 2006). Since radio powerful AGNs have largely been avoided by
  IFU surveys, it is likely that they have ignored one of the most
  important drivers of AGN feedback. The focus of our proposal is to
  address this by using MUSE to conduct the first IFU survey of a
  representative sample of local, radio-powerful AGNs.}
  
  \ImmediateObjective{

    We will use the MUSE IFU to obtain deep spectral coverage of the
    host galaxies of a representative sample of powerful radio AGNs in
    the local Universe. {\bf With these observations, we will
      determine (a) how powerful radio AGNs are triggered; (b) the role
      of jets vs. radiative driving in powering AGN outflows; and (c)
      the mass/momentum/energy content of these outflows, and whether
      they have a significant impact on their host galaxies.}
    
  \vspace{1.0mm}
  
  {\bf \large The Sample}

  Our target sample is selected from the {\bf 2~Jy sample} of radio
  powerful AGNs, originally described in Wall \& Peacock (1985). The
  original 2~Jy sample contains all sources with 2.7~GHz fluxes above
  2~Jy, although we select only the southern sample, defined as having
  declinations below $+$10\deg. This represents a total of 88 objects,
  although 42 of these have strongly beamed emission along our line of
  sight and are not suitable for studies of the host galaxy or
  resolved outflows. This leaves a sample of 46 non-beamed, southern
  2~Jy sources. From these 46, we select only those 36 at $z<0.4$
  (hereafter, ``our sample''). These 36 radio AGNs span the radio
  luminosity range:
  $25.0<{\rm log}(L_{\rm 5GHz}/{\rm W~Hz^{-1}})<27.4$. They represent
  the full diversity of radio AGNs: 36\% are broad line (i.e., Type 1)
  AGNs or quasars, 42\% are narrow-line (i.e., Type 2) AGNs, while the
  remaining 22\% are weak-line radio AGNs associated with low AGN
  accretion rates, yet still display powerful radio jets (Tadhunter et
  al. 1993). Their diversity in class and luminosity means it is
  important that we observe the full sample to obtain statistically
  meaningful numbers of AGNs in important class and luminosity
  bins. Over the past two decades, our team has led numerous campaigns
  to obtain extensive coverage of the southern 2~Jy sample across the
  full observable electromagnetic spectrum. The 2Jy is unique in terms
  of the completeness of its multi-wavelength data that includes X-ray
  imaging/spectroscopy ({\it Chandra}, XMM), optical spectroscopy (ESO
  3.6m \& VLT), optical imaging ({\it Gemini}), near-IR imaging (ESO
  NTT \& VLT), mid to far-IR photometry ({\it Spitzer} \& {\it
    Herschel}), mid-IR spectroscopy ({\it Spitzer}) and radio imaging
  (VLA and ATCA); {\it it is now the best observed of any sample of
    radio-loud AGN}.  {\bf The 2~Jy sample therefore represents our
    best possible opportunity to determine what processes trigger
    powerful radio AGNs and drive their outflows.}
  
  \vspace{1.0mm}
  
  {\bf \large The Observations}
  
  In this section we highlight how our MUSE observations of the 2~Jy
  sample will build upon our current understanding of radio AGNs,
  enabling us to determine how radio AGNs are triggered, how their
  outflows are driven, and measure the properties and impact of these
  outflows.

  % Radio AGN fall into two broad categories according to their optical
  % emission lines: {\bf Strong Line Radio Galaxies (SLRGs)} and {\bf
  %   Weak Line Radio Galaxies (WLRGs)}. SLRGs are the radio-loud
  % versions of luminous quasars, and those within the 2~Jy sample
  % represent some of the most powerful AGNs in the local Universe (see
  % review by Tadhunter et al. 2016). Studies of SLRGs therefore provide
  % insights into the triggering of powerful AGNs {\it in general}. By
  % contrast, WLRGs show little evidence of a ``traditional'' active
  % nucleus at optical wavelengths. Yet, WLRGs transfer vast amounts of
  % mechanical energy into their surrounding IGM. The 2~Jy sample
  % contains both SLRGs ({\bf \color{red} numbers}) and WLRGs ({\bf
  %   \color{red} numbers}) important to determine the root causes of
  % each mode of AGN feedback.
  
  \vspace{1.0mm}
  
  {\bf How are radio AGNs triggered?:} To establish what triggers
  radio AGNs demands detailed observations of their immediate
  surroundings, and in particular their host galaxies. Initial
  inspection reveals that the majority of radio AGNs reside in massive
  ($>10^{11}~{\rm M_\odot}$), early type galaxies (e.g., Matthews,
  Morgan \& Schmidt 1964). However, our own detailed morphological
  studies of the 2~Jy sample have revealed that features such as tidal
  tails, fans, shells, and bridges occur far more frequently around powerful
  radio AGNs compared to luminosity-matched comparison early-type
  galaxies (94\% of radio AGNs vs 50\% of non-active early-types when
  of similar surface brightness; Ramos Almeida et al. 2011, 2012;
  Fig. 2). This indicates a far more violent {\it recent} merger
  history than their early-type morphologies initially suggest.

  \vspace{1.0mm}

  The prevalence of tidal features around radio AGNs provides clear
  evidence that galaxy interactions play a role in triggering at least
  some powerful radio AGNs. However, with both major and minor mergers
  capable of producing strong tidal features, it is not clear what
  type of merger trigger radio AGN. Thankfully, stellar kinematics can
  be used to distinguish between the types of mergers that have taken
  place. Fast rotating early-type galaxies are produced by wet, major
  mergers, while slow rotators are thought to be produced by dry major
  or a series of minor mergers (Cappellari et al. 2016 and references
  therein).

  \vspace{1.0mm}

  To date, only a handful of radio AGNs have had their hosts' stellar
  kinematics measured, and even then only with long-slit spectra out
  to only a fraction of an effective radius (Smith and Heckman et
  al. 1990, Bettoni et al. 2001). These few shallow observations
  provide tantalising, yet unconfirmed, evidence that radio AGNs are
  associated with fast rotators. If confirmed, this would connect
  powerful radio AGNs to major wet mergers, possibly representing a
  post-starburst phase, since local Ultra Luminous Infrared Galaxies
  (ULIRGs) are also associated with fast rotation (e.g., Genzel et
  al. 2001; Tacconi et al. 2002). Should they instead be triggered by
  an alternative mechanism, such as accretion of hot halo gas or minor
  mergers as suggested by some models (e.g., King \& Pringle 2006;
  Hopkins \& Quataert 2010), they would show a preference toward
  slowly-rotating early-type galaxies. Finally, should radio AGNs show
  no preference to either fast or slow rotators, it would imply that
  the type of merger is not important, with either wet, dry, major or
  minor capable of triggering them. {\bf By mapping the stellar
    kinematics across the entire host galaxy, our MUSE observations
    will determine what type of galaxy interaction are required to
    trigger powerful radio AGNs.}

  \vspace{1.0mm}

  With our MUSE observations, we will spatially resolve (to at least
  one effective radius) the host galaxies of all our sample. We will
  measure the off-nuclear stellar velocity shift ($V$) and velocity
  dispersion ($\sigma$) from stellar absorption lines. At the low
  redshifts of these sources, the MUSE spectra will cover the strong
  Mg{\it b}$\lambda$5200 absorption line, although we will fit the
  whole stellar continuum with gaussian-convolved stellar templates to
  maximise the information from the stellar continuum (i.e., excluding
  emission lines and using the Penalised Pixel-Fitting method of
  Cappellari \& Emsellem, 2004). To extract just the first two
  velocity modes (i.e., $V$ and $\sigma$) requires a comparitavely low
  continuum signal-to-noise of $\sim$10 (e.g., Cappellari et
  al. 2007). To ensure we measure the characteristic kinematics of the
  whole galaxy, rather than just the core, {\it it is vital that we
    reach this sensitivity to at least one effective radius} (a
  diameter of 20 kpc, or 20\arcsec\ [4\arcsec ] for our lowest
  [highest] redshift target). This can only be achieved with deep
  ($\sim2~{\rm hr}$) observations (see Box 9).

  \vspace{1.0mm}
  
  We will use the criteria of Emsellem et al. (2007) to distinguish
  between fast and slow rotating early type radio AGN hosts. This uses
  the resolved $V$ and $\sigma$ maps to overcome some of the
  degeneracies associated with traditional $V/\sigma$
  measurements. Next, we will compare the relative numbers of fast and
  slow rotators in our sample against mass-matched samples of non-AGN
  early type galaxies from the {\sc Atlas}$^{\rm 3D}$ survey
  (Cappellari et al. 2011). Should our sample display a significant
  departure from the relative numbers of fast to slow rotators in this
  comparison sample, it would imply a preference of one over the
  other. We will also compare against the radio-weak AGNs targeted with
  MUSE as part of the Close AGN Reference Survey (CARS; PI: Husemann)
  to determine whether powerful radio AGNs show evidence of being
  triggered via different mechanisms compared to radio-weak AGNs. With
  36 radio AGNs, our sample contains sufficient statistics to robustly
  test for this as a function of AGN type (broad line, narrow line,
  weak line) and luminosity, thereby demonstrating whether different
  classes of AGNs are triggered by different mechanisms.

  \vspace{1.0mm}
  
  {\bf What drives outflows from radio AGNs?:} To have any impact on
  galaxy evolution, the energy from an AGN, radio powerful or
  otherwise, must be transmitted into their host galaxies or
  surrounding material {\it and} effect some influence. Our team has
  played a leading role in identifying fast, yet non-relativistic
  (${\rm \sim1000~km\ s^{-1}}$) winds associated with powerful AGNs
  (e.g., Tadhunter et al. 2001, 2014; Alexander et al. 2008;
  Villar-Martin et al. 2014, 2016; Harrison et al. 2014; Santoro et
  al. 2015; Spence et al. 2016; Figs. 3 \& 4). These winds are often
  extended over kpc-scales, thereby {\it potentially} affecting a
  large fraction of the host galaxy as demanded by AGN feedback models
  (e.g., Alexander et al, 2008; Harrison et al. 2012, 2014; Tadhunter
  et al. 2014). Our investigations have also found that such winds are
  more prevalent among AGNs displaying evidence of nuclear radio
  emission.  Indeed, our long-slit observations of a subsample of the
  2~Jy sample has shown that some of the fastest AGN winds in the
  local Universe are associated with powerful radio AGNs (e.g.,
  Tadhunter et al. 2016). As such, it is important for our models of
  feedback that we establish their primary driving mechanism, whether
  the radio jet or the intense radiation from the AGN. {\bf Our MUSE
    observations will achieve this by mapping the winds in the 2~Jy
    AGN, allowing us to relate them to the resolved radio jets.}

  \vspace{1.0mm}

  By measuring the profiles of the strong forbidden $[$O~{\sc
    iii}$]\lambda5007$ emission line, we will map the kinematics of
  the ionised gas in our sample. The $[$O~{\sc iii}$]$ line traces low
  density gas and has been used extensively in recent studies, not
  least our own, to successfully measure the kinematics and extent of
  outflows in AGN host galaxies. The ionised phase, however, only
  represents a fraction of the gas in a galaxy, so we will use the
  Na~{\sc i} $\lambda\lambda5890,5896$ absorption lines to map the
  kinematics of the neutral gas phase (e.g., Rupke \& Veilleux,
  2013). {\it It is important that we have sufficiently deep
    observations to measure this absorption line against the stellar
    continuum}.  Any Na~{\sc i} absorption intrinsic to the stellar
  continuum will be modelled using stellar templates. At every
  position in the galaxy, we will decompose the $[${\rm O}~{\sc
    iii}$]$ and Na~{\sc i} lines into their various kinematic
  components, identifying those regions displaying the strongly
  shifted or broad kinematic components characteristic of powerful AGN
  outflows. By mapping the stellar and gas kinematics across the whole
  galaxy, {\it our deep observations will be able to identify even
    small departures from gravitational motions, and thus be highly
    sensitive to extended outflows}. We will then relate the location
  of the outflows to the positions of the radio jets that have already
  been mapped-out at high (i.e., sub-arcsecond) resolutions with our VLA
  and ATCA observations. {\it Should any AGN show outflows with a
    significantly wider opening angle than the jets, it would imply
    radiative driving in that case.}
  
  \vspace{1.0mm}
  
  {\bf What effect do AGN outflows have on their host galaxies?:} As
  well as establishing the mechanism driving AGN winds, to test AGN
  feedback models we must also determine whether they have any real
  effect on their host galaxies. The first step in doing this is to
  measure the energy content of the wind and (i) compare it to model
  predictions and (ii) empirically assess whether it is sufficient to
  evacuate the galaxy of significant fraction of its gas content. The
  second step is to determine whether the winds are having a direct
  influence on the host by comparing the rates of star formation
  within the winds against those in the rest of the galaxy. All of the
  2~Jy sample have {\it Spitzer} and {\it Herschel} coverage,
  providing obscuration-independent star formation rates
  (SFRs). However, the host is only resolved at these infrared
  wavelengths in a small minority of cases. For the others, we will
  use the MUSE spectra to produce resolved BPT diagnostics to map-out
  regions of star-formation within the host galaxies and, following
  e.g., Perna et al. (2015) \& Maiolino et at. (2017), relate these
  regions to the resolved AGN outflows. {\bf By mapping the source of
    ionising radiation (AGN, star formation), and density and
    kinematics of the outflows, our MUSE observations will provide the
    information needed to comprehensively measure the impact of radio
    AGNs on their host galaxies.}

  \vspace{1.0mm}  
  
  To measure the energy content of the extended outflows requires
  knowledge of their mass content. Thirty-five of our sample are at
  low enough redshifts that the $[$S~{\sc
    ii}$]\lambda\lambda6717,6732$ doublet falls within the spectral
  range of MUSE, enabling us to use this density-sensitive line for
  mass measurements. This density information, combined with the
  kinematics and extents of the outflows, will enable us to determine
  the mass and mass flux of the outflows. From this, we will measure
  the energy and momentum content of the outflows. This will be
  compared against model predictions (e.g., Zubovas \& King 2012;
  Gabor \& Bournaud 2014) and the gravitational potential of the
  galaxy (measured from stellar and non-outflowing gas kinematics) to
  determine whether the outflows contain sufficient energy to have a
  significant impact on host.

  \vspace{1.0mm}  
  
  Finally, we will use the approach of using resolved BPT diagnostics
  outlined in e.g., Perna et al. (2015); Maiolino et al. (2017) to
  directly measure the impact of the outflows on star formation within
  the extended outflows. This method solves the problem that the
  H$\beta$, $[$O~{\sc iii}$]\lambda5007$, $[$O~{\sc
    iii}$]\lambda\lambda6548, 6584$ and H$\alpha$ lines (i.e., the
  traditional emission lines used in BPT diagnostics) are dominated by
  AGN ionisation by kinematically decomposing the emission lines into
  separate AGN-ionised and star formation-ionised components. By
  relating regions of star-formation to the locations of the outflows,
  we will measure the impact of these outflows on star formation
  within the host galaxy (whether they enhance [positive feedback] or
  suppress [negative feedback] star-formation). Since BPT diagnostics
  are sensitive to star formation within the past $\sim$10~Myr, this
  provides a measurement of the immediate effect of AGN feedback
  (whether positive or negative) on short timescales before the AGN
  has had a chance to ``switch off'' (e.g., Hickox et al. 2014)}

\TelescopeJustification{MUSE is the only IFU available to us that has
  the sensitivity and field-of-view needed to achieve our immediate
  science goals.}

\ModeJustification{As we are applying to use MUSE in a standard setup,
  service mode is more suitable as it guarantees our requested
  observing conditions.}

%%%%%%%%%%%%%%%%%%%%%%%%%%%%%%%%%%%%%%%%%%%%%%%%%%%%%%%%%%%%%%%%%%%%%%%%
%%%%% THE TWO PAGES OF THE FIGURES %%%%%%%%%%%%%%%%%%%%%%%%%%%%%%%%%%%%%
%%%%%%%%%%%%%%%%%%%%%%%%%%%%%%%
%%%%%%%%%%%%%%%%%%%%%%%%%%%%%%%%%%%%%%%%%
%
% Up to TWO pages of figures can be added to your proposal.  If you
% use color figures, make sure that they are still readable if printed
% in black and white.  Figures must be in PDF or JPEG format.
% Each figure has a size limit of 1MB.


\MakePicture{Figures1.pdf}{angle=0,width=18cm}
\MakePicture{Figures2.pdf}{angle=0,width=18cm}
%\MakeCaption{Fig.~1: A caption for your figure can be inserted here.}

%%%%%%%%%%%%%%%%%%%%%%%%%%%%%%%%%%%%%%%%%%%%%%%%%%%%%%%%%%%%%%%%%%%%%%%%
%%%%% THE PAGE OF BOXES 6, 7, AND 8 %%%%%%%%%%%%%%%%%%%%%%%%%%%%%%%%%%%%
%%%%%%%%%%%%%%%%%%%%%%%%%%%%%%%%%%%%%%%%%%%%%%%%%%%%%%%%%%%%%%%%%%%%%%%%
%
%---- BOX 6 ------------------------------------------------------------
%
% Indicate below the experience of the applicants with telescopes,
% instrumentation, data reduction and delivery of data products to the 
% ESO Archive.
% This macro is NOT checked at the pdfLaTeX compilation.

% Indicate here the experience of the applicants with telescopes,
% instrumentation, data reduction and their track record for submission
% of data products to the ESO Archive from previous large
% programmes. The PI should describe the data quality assessment process
% and the data reduction for the production of science data products.

\Experience{

  The members of our team have extensive experience using the VLT and,
  in particular, its IFUs. Together, we have reduced and published
  data from KMOS, MUSE, VIMOS, and SINFONI representing over 100 hours
  of observations. A significant proportion of the KMOS data was taken
  as part of an ongoing Guaranteed Time Project (led by co-I
  D. M. Alexander). The first results from this GT program have
  already been published (Harrison et al. 2016, 2017) and the first
  results of the analyses have been uploaded to the Vizier
  database. We note that members of our team (Santoro, Morganti) were
  awarded Science Verification time on MUSE; the resulting data has
  already been fully reduced, analysed and published (Santoro et
  al. 2015ab, 2016). For the proposed MUSE observations, we have
  allocated a team with extensive experience of ESO pipelines and IFU
  data reduction (Harrison, Mullaney, Rose, Santoro) to perform an
  initial reduction within 24 hours of being notified of the
  completion of each OB. The data will be reduced using the standard
  ESO pipeline on a dedicated 52-core, 256~GB RAM machine with
  off-site backup. Since each OB contains three exposures, we will
  immediately be able to check for spurious outliers within each
  OB. Furthermore, as each OB represents a third of all data to be
  taken on each object, there will be sufficient signal per OB to
  check whether the desired data quality has been obtained, including
  seeing, signal-to-noise, and photometric reliability. As such, we
  have the capacity to notify the support staff of any problems with
  the data before significant amounts of further observations are
  made. After the initial checks have been completed, we will perform
  at least two full independent data reductions for each OB; one based
  at Sheffield (Mullaney, Rose), and another based either at ESO
  (Harrison) or ASTRON (Santoro). Rose, Harrison and Santoro, in
  particular, have significant experience manipulating the ESO data
  reduction pipeline to optimise outputs, so each reduction will be
  truly independent. Thus, we will be able to perform consistency
  checks and select the optimum output prior to submission to the
  archive. As well as reduced datacubes, we will also submit the
  results of our analyses to the ESO archive. These analysis products
  will include the results extracted from fitting the stellar
  continuum and the emission line profiles. Finally, the PI has a
  track record of making his analysis scripts public, a legacy that
  will continue with this program, facilitating reproducability checks
  and the easy analysis of the public datacubes by others.
  
}

%---- BOX 7 ------------------------------------------------------------
%
% Indicate below the strategy for data reduction and analysis with
% description of the available resources to the observing team, such
% as: computing capabilities, research assistants, etc...
% In addition a delivery plan for data products should be described here.
% This macro is NOT checked at the pdfLaTeX compilation.

\Resources{

  We have allocated a team of four full-time staff and PDRAs to data
  reduction, all of whom have significant experience with using the
  ESO data reduction pipeline to reduce IFU data. In terms of
  computing resources, the PI's group has access to a dedicated
  52-core, 256~GB RAM machine for data reduction. Data analysis will
  cover stellar continuum and emission line fitting across the
  IFU. This analysis will be carried-out using publicly available
  software (e.g., pPXF, Cappellari \& Emsellem 2004 ), as well as our
  own in-house IDL and Python software. Reduced (but not analysed)
  data from each semester will be published within six months of the
  end of that semester.  As well as the reduced data, the PI will
  ensure that all results of this data analyses are published to the
  ESO archives within two years of the completed program. Furthermore,
  all in-house software used to conduct these analyses will be made
  publicly available. This will enable others to both check our
  results and easily perform their own analyses using alternative
  input parameters or on, for example, other emission lines not
  analysed by default. In what follows, we highlight the tasks that
  each member of the team has been allocated in order to ensure that
  all steps in the data reduction, analyses and science publication
  are carried-out within the required four-year timeframe (two years'
  of observations $+$ two years of analyses and science publications):

  \vspace{1mm}

  {\it Data quality assessment:} Mullaney (S), Harrison (F), Rose (P),
  Santoro (D)

  {\it Data reduction:} Mullaney (S), Harrison (F), Rose (P) , Santoro (D)

  {\it Data analyses (i.e., spectral fitting etc):} Mullaney (S),
  Harrison (F), Rosario (F), Rose (P), Spence (D)

  {\it Host galaxy and kinematics:} Tadhunter (S), Ramos-Almeida (S),
  Bessiere (F), Dicken (S), Bernhard (P)

  {\it Radio Jet/Outflows:} Mullaney (S), Tadhunter (S), Morganti (S),
  Santoro (D), Spence (D)

  {\it Outflow properties:} Alexander (S), Harrison (F), Villar-Martin
  (S), Rose (P), Tadhunter (S)

  {\it Outflow/host interaction:} Mullaney (S), Alexander (S), Harrison (F),
  Rosario (F), Villar-Martin (S), Bessiere (F)

  {\it Key:} S $=$ Staff; F $=$ Research Fellow; P $=$ PDRA; D $=$ Doctoral Student

}

%---- BOX 8 ------------------------------------------------------------
%
% Take advantage of this box to provide any special remark (up to ten
% lines).  
% This macro is NOT checked at the pdfLaTeX compilation.

\SpecialRemarks{This is a resubmission of a P99 LP proposal. In that
  previous proposal, we requested observations of the full Southern
  (unbeamed) 2~Jy sample of 46 AGNs spanning a redshift range
  $0.05<z<0.7$. However, while considering the science case ``strong
  and convincing'', the panel raised concerns regarding the loss of
  linear resolution in the more distant targets. Acknowledging these
  concerns, we now restrict the request to target only the $z<0.4$
  AGNs to ensure that we can reach our science goals for the entire
  sample, thus maximising science output per observing-hour. On the
  request of the OPC, we now also provide the redshifts and radio
  luminosities of all our targets (see Box 12). To address another of
  the OPC's comments: we stress that there is no better studied sample
  of radio luminous AGNs {\it in the whole sky} than the 2~Jy
  sample. It is therefore the optimal sample for determining the
  causes and effects of radio AGNs.}

%%%%%%%%%%%%%%%%%%%%%%%%%%%%%%%%%%%%%%%%%%%%%%%%%%%%%%%%%%%%%%%%%%%%%%%%
%%%%% THE PAGE OF BOXES 9 AND 10 %%%%%%%%%%%%%%%%%%%%%%%%%%%%%%%%%%%%%%%
%%%%%%%%%%%%%%%%%%%%%%%%%%%%%%%%%%%%%%%%%%%%%%%%%%%%%%%%%%%%%%%%%%%%%%%%
%
%---- BOX 9 ------------------------------------------------------------
%
% Provide below a careful justification of the requested lunar phase
% and of the requested number of nights or hours.  
% All macros in this box are NOT checked at the pdfLaTeX compilation.

\WhyLunarPhase{Since measuring the stellar kinematics and outflow
  properties (via $[$O~{\sc iii}$]\lambda5007$) relies on the
  detection of features in the bluer region of the MUSE spectrum, we
  require dark or grey skies.}

% Provide here a careful justification of the requested number of nights
% or hours.  ESO Exposure Time Calculators exist for all Paranal and La
% Silla instruments and are available at the following web address:
% http://www.eso.org/observing/etc .  Links to exposure time calculators
% for APEX instrumentation can be found in Section 7 of the Call for
% Proposals.

\WhyNights{The integration time per target is dictated by the need to
  obtain sufficiently high signal-to-noise in both the stellar
  continuum and $[$O~{\sc iii}$]\lambda5007$ emission line to map
  their kinematics to at least one effective radii
  ($\sim10$~kpc). This physical extent is needed to (a) ensure that
  any measured stellar kinematics provide a good representation of the
  whole galaxy, not just the (possibly decoupled) core and (b) measure
  the full sphere of influence of any outflows and assess their impact
  on a significant fraction of the host. Based on previous experience,
  this demands a signal-to-noise per spectral bin of at least 10 in
  the continuum. From our own optical imaging campaigns of the 2~Jy
  sample, the v-band surface brightness of the host at one effective
  radii is typically $\sim24^{\rm th}$~mag arcsec$^{-2}$ (Vega
  mags.). To obtain a S/N of 10 per pixel at this surface brightness
  will require unfeasibly long integration times. However, we will
  adaptively spatially bin to obtain the desired S/N. Entering this
  surface brightness into the MUSE ETC, adopting an elliptical
  spectral template, three days from new moon, 1.0\arcsec\ seeing (to
  sufficiently resolve the hosts of all AGNs in our sample) and a
  2-pixel spectral binning results in a S/N of 1.5 per pixel with
  2.5-hours on-target. Binning $6\times6$ pixels will achieve a S/N of
  $\sim10$, without a significant loss of spatial resolution beyond
  the requested 1.0\arcsec\ seeing. We will split each 2.5-hour
  on-target observation into twelve 750s DITs, with four DITs per OB
  (50 mins per OB). Following the MUSE instrument guidelines on
  overheads, we allocate 12~min per OB to overheads (pointing,
  acquisition, readout, etc.), resulting in 62~mins per OB. One of our
  sample has already been observed with MUSE for 1hr 20min, and
  therefore requires just two OBs. With three OBs for the remaining 35
  targets, this gives a total time request of 110.6 hours, distributed
  over four semesters.}

% Please specify the type of calibrations needed.
% Valid values: standard, special
% In case of special calibration the second parameter specifies them

%\Calibrations{special}{Adopt a special calibration}
\Calibrations{standard}{}


%---- BOX 10 -----------------------------------------------------------
% 
% Use of the ESO facilities during the last 2 years (4 observing
% periods) and description of the status of the obtained data.
% This macro is NOT checked at the pdfLaTeX compilation.

\LastObservationRemark{

{\bf KMOS; 095.A-748; 096.A-200; 097.A-0182; 098.A-311} Ongoing GTO program to measure
the dynamics, star formation, outflows etc. of star-forming galaxies
and AGN. First papers: {\bf {\em Stott et al. (2016); Harrison
et~al. (2016b, 2017); Magdis et~al. (2016); Tiley et al. (2016); Swinbank et al. 2017}}.

{\bf ALMA; 2016.1.00735.S} ``Spatially-resolved star formation at
high-z; are AGN host galaxies special?''. Observations open.

{\bf SINFONI; 096.A-600} ``AO/IFU observations of a representative
sample of $z\approx1.5$ AGN: what influence do AGN have on their host
galaxies?''; Observations carried over, 50\% completed.

{\bf VIMOS; 099.B-0793}. ``Quasars ``Blowing Bubbles'': the impact of AGN on host galaxies'', in queue.

}
%---- BOX 11 -----------------------------------------------------------
%
% Applicant's publications related to the subject of this proposal
% during the past two years.  Use the simplified abbreviations for
% references as in A&A.  Separate each reference with the following
% usual LaTex command: \smallskip\\
%   
%   Name1 A., Name2 B., 2001, ApJ, 518, 567: Title of article1
%   \smallskip\\
%   Name3 A., Name4 B., 2002, A\&A, 388, 17: Title of article2
%   \smallskip\\
%   Name5 A. et al., 2002, AJ, 118, 1567: Title of article3
%
% This macro is NOT checked at the pdfLaTeX compilation.

\Publications{

  Harrison, C.~M., 2017b, Nature Astronomy, arXiv:1703.06889, Invited
  Review: {\it Impact of supermassive black
    hole growth on star formation'}\\
  Harrison, C.~M., et~al. 2017a, MNRAS, 467, 1965, ``The KMOS Redshift
  One Spectroscopic Survey (KROSS): rotational velocities and angular
  momentum of z~0.9 galaxies''\\
Morganti et al. 2016, A\&A, 593, 30: {\it Another piece of the
    puzzle: The fast H I outflow in Mrk 231}\\
  Morganti et al. 2016, A\&A, 592, 94: {\it Missing link: Tracing
    molecular gas in the outer filament of Centaurus A}\\
  Johnson, Harrison et al. 2016, MNRAS, 460, 1059: {\it The spatially resolved
    dynamics of dusty starburst galaxies in a z ˜ 0.4 cluster:
    beginning the transition from spirals to S0s}\\
  Villar-Martin et al. 2016, MNRAS, 460, 130: {\it Ionized outflows in luminous
    type 2 AGNs at z < 0.6: no evidence for significant impact on the
    host galaxies}\\
  Spence et al. 2016, MNRAS, 459, 16: {\it No evidence for large-scale outflows
    in the extended ionized halo of ULIRG Mrk273}\\
  Tadhunter 2016, A\&AR, 24, 10: {\it Radio AGN in the local universe:
    unification, triggering and evolution}\\
  Santoro et al. 2016, A\&A, 590, 37S: {\it Embedded star formation in
    the extended narrow line region of Centaurus A: Extreme mixing
    observed by MUSE}\\
  Villar-Martin et al. 2015, MNRAS, 454, 419: {\it Deconstructing the
    narrow-line region of the nearest obscured quasar}\\
  Morganti et al. 2015, A\&A, 580, 1: {\it The fast molecular outflow
    in the Seyfert galaxy IC 5063 as seen by ALMA}\\
  Santoro et al. 2015, A\&A, 575, 4: {\it The outer filament of
    Centaurus A as seen by MUSE}\\
  Harrison et al. 2015, ApJ, 800, 45: {\it Storm in a ``Teacup'': A
    Radio-quiet Quasar with ≈10 kpc Radio-emitting Bubbles and Extreme
    Gas Kinematics}\\
  Santoro et al. 2015, A\&A, 574, 89: {\it The jet-ISM interaction in
    the outer filament of Centaurus A}\\
  Tadhunter et al. 2014, MNRAS, 445, 51: {\it The dust masses of
    powerful radio galaxies: clues to the triggering of their
    activity}\\
}

%%%%%%%%%%%%%%%%%%%%%%%%%%%%%%%%%%%%%%%%%%%%%%%%%%%%%%%%%%%%%%%%%%%%%%%%
%%%%% THE PAGE OF THE TARGET/FIELD LIST %%%%%%%%%%%%%%%%%%%%%%%%%%%%%%%%
%%%%%%%%%%%%%%%%%%%%%%%%%%%%%%%%%%%%%%%%%%%%%%%%%%%%%%%%%%%%%%%%%%%%%%%%
%
%---- BOX 12 -----------------------------------------------------------
%
% Complete list of targets/fields requested.  The macro takes nine 
% parameters: run ID, target field/name, RA, Dec, time on target, magnitude, diameter,
% additional information, reference star.
%
% 1. RUN ID
% Valid values: run IDs specified in BOX 3
% 
% 2. Target field/Name
%
% 3. RA (J2000)
% Format: hh mm ss.f, or hh mm.f, or hh.f
% Use 00 00 00 for unknown coordinates
% This parameter is NOT checked at the pdfLaTeX compilation.
% 
% 4. Dec (J2000)
% Format: dd mm ss, or dd mm.f, or dd.f
% Use 00 00 00 for unknown coordinates
% This parameter is NOT checked at the pdfLaTeX compilation.
%
% 5. TIME ON TARGET
% Format: hours (overheads and calibration included)
% This parameter is NOT checked at the pdfLaTeX compilation.
%
% 6. MAGNITUDE
% This parameter is NOT checked at the pdfLaTeX compilation.
%
% 7. ANGULAR DIAMETER
% This parameter is NOT checked at the pdfLaTeX compilation.
%
% 8. ADDITIONAL INFORMATION
% Any relevant additional information may be inserted here.
% For APEX runs, the requested PWV and the acceptable LST range
%     MUST be specified here for each target. 
% This parameter is NOT checked at the pdfLaTeX compilation.
%
% 9. REFERENCE STAR ID
% See Users' Manual.
% This parameter is NOT checked at the pdfLaTeX compilation.
%
% Long lists of targets will continue on the last page of the
% proposal.
%
%
%                       ** VERY IMPORTANT ** 
% The scheduling of your programme will take into account ALL targets
%
% DO NOT USE ANY TEX/LATEX MACROS FOR THE TARGETS

\Target{A}{PKS0347+05}{03 49 46.5}{+05 51 42.3}{3.1}{26.7}{}{z=0.34}{}
\Target{A}{PKS0404+03}{04 07 16.5}{+03 42 25.8}{3.1}{25.7}{}{z=0.09}{}
\Target{A}{PKS0442-28}{04 44 37.7}{-28 09 54.6}{3.1}{26.1}{}{z=0.15}{}
\Target{A}{PKS0625-35}{06 27 06.7}{-35 29 16.3}{3.1}{25.2}{}{z=0.05}{}
\Target{A}{PKS0806-10}{08 08 53.6}{-10 27 40.2}{3.1}{25.7}{}{z=0.11}{}
\Target{A}{PKS0915-11}{09 18 05.7}{-12 05 44.0}{3.1}{26.0}{}{z=0.05}{}
\Target{A}{PKS0620-52}{06 21 43.3}{-52 41 33.3}{3.1}{24.9}{}{z=0.05}{}
\Target{B}{PKS0034-01}{00 37 04.1}{-01 09 08.2}{3.1}{25.3}{}{z=0.07}{}
\Target{B}{PKS0043-42}{00 46 17.8}{-42 07 51.4}{3.1}{26.0}{}{z=0.12}{}
\Target{B}{PKS1934-63}{19 39 25.0}{-63 42 45.6}{2.1}{26.8}{}{z=0.18}{}
\Target{B}{PKS2356-61}{23 59 04.5}{-60 54 59.1}{3.1}{26.0}{}{z=0.10}{}
\Target{B}{PKS1839-48}{18 43 14.6}{-48 36 23.3}{3.1}{25.6}{}{z=0.11}{}
\Target{B}{PKS1733-56}{17 37 35.8}{-56 34 03.4}{3.1}{25.9}{}{z=0.10}{}
\Target{B}{PKS2250-41}{22 53 03.2}{-40 57 46.2}{3.1}{26.6}{}{z=0.31}{}
\Target{B}{PKS2211-17}{22 14 25.8}{-17 01 36.2}{3.1}{26.1}{}{z=0.15}{}
\Target{B}{PKS0038+09}{00 40 50.5}{+10 03 26.8}{3.1}{26.2}{}{z=0.19}{}
\Target{B}{PKS2135-14}{21 37 45.2}{-14 32 55.5}{3.1}{26.2}{}{z=0.20}{}
\Target{B}{PKS1949+02}{19 52 15.8}{+02 30 23.1}{3.1}{25.3}{}{z=0.06}{}
\Target{C}{PKS1151-34}{11 54 21.8}{-35 05 29.1}{3.1}{26.7}{}{z=0.26}{}
\Target{C}{PKS0945+07}{09 47 45.2}{+07 25 20.4}{3.1}{25.7}{}{z=0.09}{}
\Target{C}{PKS0105-16}{01 08 16.9}{-16 04 20.6}{3.1}{26.8}{}{z=0.40}{}
\Target{C}{PKS0859-25}{09 01 47.5}{-25 55 19.0}{3.1}{26.7}{}{z=0.31}{}
\Target{C}{PKS0349-27}{03 51 35.8}{-27 44 33.8}{3.1}{25.3}{}{z=0.07}{}
\Target{C}{PKS0625-53}{06 26 20.4}{-53 41 35.2}{3.1}{25.0}{}{z=0.05}{}
\Target{C}{PKS0213-13}{02 15 37.5}{-12 59 30.5}{3.1}{26.0}{}{z=0.15}{}
\Target{D}{PKS1559+02}{16 02 27.4}{+01 57 55.7}{3.1}{25.9}{}{z=0.10}{}
\Target{D}{PKS0035-02}{00 38 20.5}{-02 07 40.1}{3.1}{26.2}{}{z=0.22}{}
\Target{D}{PKS0039-44}{00 42 09.0}{-44 14 01.3}{3.1}{26.7}{}{z=0.35}{}
\Target{D}{PKS1355-41}{13 59 00.2}{-41 52 54.1}{3.1}{26.7}{}{z=0.31}{}
\Target{D}{PKS0023-26}{00 25 49.2}{-26 02 12.8}{3.1}{27.0}{}{z=0.32}{}
\Target{D}{PKS1648+05}{16 51 08.2}{+04 59 33.8}{3.1}{26.9}{}{z=0.15}{}
\Target{D}{PKS1814-63}{18 19 35.0}{-63 45 48.1}{3.1}{25.5}{}{z=0.06}{}
\Target{D}{PKS1932-46}{19 35 56.6}{-46 20 40.7}{3.1}{26.7}{}{z=0.23}{}
\Target{D}{PKS1954-55}{19 58 16.1}{-55 09 37.5}{3.1}{25.1}{}{z=0.06}{}
\Target{D}{PKS2221-02}{22 23 49.6}{-02 06 12.4}{3.1}{25.2}{}{z=0.06}{}
\Target{D}{PKS2314+03}{23 16 35.2}{+04 05 18.2}{3.1}{26.3}{}{z=0.22}{}


%                      *****************
%                      ** PWV limits **
% For all APEX instruments users must specify the PWV upper
% limits for each target. For example:
%\Target{}{Alpha CMa}{06 45 08.9}{-16 42 58}{1}{-1.4}{6 mas}{PWV=1.0mm, Sirius}{}
%\Target{}{HD 104237}{12 00 05.6}{-78 11 33}{1}{}{}{PWV<0.7mm;LST=9h00-15h00}{}
%                      *****************

% Use TargetNotes to include any comments that apply to several or all
% of your targets.
% This macro is NOT checked at the pdfLaTeX compilation.

\TargetNotes{The ``Mag.'' column contains the $\rm log_{10}$
  rest-frame 5~GHz radio luminosity of the target in units of
  ${\rm W~Hz^{-1}}$.}


%%%%%%%%%%%%%%%%%%%%%%%%%%%%%%%%%%%%%%%%%%%%%%%%%%%%%%%%%%%%%%%%%%%%%%%
%---- BOX 12a -- ESO Archive ------------------------------------------
%%%%%%%%%%%%%%%%%%%%%%%%%%%%%%%%%%%%%%%%%%%%%%%%%%%%%%%%%%%%%%%%%%%%%%%
% Are the data requested in this proposal on the ESO Archive
% (http://archive.eso.org)? If yes, explain the need for new data.
% This macro is NOT checked at the pdfLaTeX compilation.

\RequestedDataRemark{One of our targets, PKS1934-063 has been observed
by MUSE for a total of 1hr20 minures. This data is not, however, deep
enough to reach our primary science goals. We therefore request a
further 100 mins observing time on this single object (to fit within
our OB structure).}

%%%%%%%%%%%%%%%%%%%%%%%%%%%%%%%%%%%%%%%%%%%%%%%%%%%%%%%%%%%%%%%%%%%%%
%---- BOX 12b -- ESO GTO/Public Survey Programme Duplications--------
%%%%%%%%%%%%%%%%%%%%%%%%%%%%%%%%%%%%%%%%%%%%%%%%%%%%%%%%%%%%%%%%%%%%%
% If any of the targets/regions in ongoing GTO Programmes and/or 
% Public Surveys are being duplicated here, please explain why.

\RequestedDuplicateRemark{

There is no duplication of the requested targets in ongoing GTO programmes.

}

%%%%%%%%%%%%%%%%%%%%%%%%%%%%%%%%%%%%%%%%%%%%%%%%%%%%%%%%%%%%%%%%%%%%%%%%
%%%%% THE PAGE OF SCHEDULING REQUIR. AND INSTRUMENT CONFIGURATIONS %%%%%
%%%%%%%%%%%%%%%%%%%%%%%%%%%%%%%%%%%%%%%%%%%%%%%%%%%%%%%%%%%%%%%%%%%%%%%%
%
%---- BOX 13 -----------------------------------------------------------
%
% 1. RUN SPLITTING, FOR A GIVEN ESO TELESCOPE (Visitor Mode only)
%

% This line should remain uncommented if the proposal involves 
% time-critical observations, or observations to be performed at specific 
% time intervals. Please leave these brackets blank. Details of time 
% constraints can be entered in Special Remarks and using the 
% other flags in Box 13.
%
%\HasTimingConstraints{}


% 1st argument: run ID
% Valid values: run IDs specified in BOX 3
%
% 2nd argument: run splitting requested for sub-runs
% This parameter is NOT checked at the pdfLaTeX compilation.

%\RunSplitting{B}{2,10s,2,20w,2}
%\RunSplitting{C}{2,10s,2,20w,2,15s,4H2}

% 2. SPECIFIC DATE(S) FOR TIME-CRITICAL OBSERVATIONS
% Please note: The dates must correspond to the LOCAL CHILEAN observing dates.
%
% 1st argument: run ID
% Valid values: run IDs specified in BOX 3
%
% 2nd argument: Chilean start date for the critical period
% Format: dd-mmm-yyyy
% This parameter is NOT checked at the pdfLaTeX compilation.
%
% 3rd argument: Chilean end date for the critical period
% Format: dd-mmm-yyyy
% This parameter is NOT checked at the pdfLaTeX compilation.
%
% 4th argument: Reason for the time-critical dates specified.
% 

%\TimeCritical{A}{12-may-17}{14-may-17}{Insert reason for time-critical observations.}
%\TimeCritical{D}{12-may-17}{14-may-17}{Insert reason for time-critical observations.}


% 3. UNSUITABLE PERIOD(S) OF TIME
%
% 1st argument: run ID
% Valid values: run IDs specified in BOX 3
%
% 2nd argument: Chilean start date for the unsuitable time
% Format: dd-mmm-yyyy
% This parameter is NOT checked at the pdfLaTeX compilation.
%
% 3rd argument: Chilean end date for the unsuitable time
% Format: dd-mmm-yyyy
% This parameter is NOT checked at the pdfLaTeX compilation.

%\UnsuitableTimes{A}{15-jul-17}{18-jul-17}{Insert explanation of unsuitable time here.}
%\UnsuitableTimes{B}{15-jul-17}{18-jul-17}{Insert explanation of unsuitable time here.}
%\UnsuitableTimes{C}{20-jul-17}{23-jul-17}{Insert explanation of unsuitable time here.}

% 4. LINK FOR COORDINATED OBSERVATIONS BETWEEN DIFFERENT RUNS.
%
% 1st argument: run ID
% Valid values: run IDs specified in BOX 3
%
% 2nd argument: relationship
% Valid value: after, simultaneous
%
% 3rd argument: run ID
% Valid values: run IDs specified in BOX 3

%\Link{B}{after}{A}{10}
%\Link{C}{after}{B}{}

%%%%%%%%%%%%%%%%%%%%%%%%%%%%%%%%%%%%%%%%%%%%%%%%%%%%%%%%%%%%%%%%%%%%%%%%
%
%---- BOX 14 -----------------------------------------------------------
%
% INSTRUMENT CONFIGURATIONS:
%
% Uncomment only the lines related to instrument configuration(s)
% needed for the acquisition of your planned observations.
%
% 1st argument: run ID
% Valid values: run IDs specified in BOX 3
%
% 2nd argument: instrument
% This parameter is NOT checked at the pdfLaTeX compilation.
%
% 3rd argument: mode
% This parameter is NOT checked at the pdfLaTeX compilation.
% Please note that RRM mode is only available for some specific
% instrument configurations. 
%
% 4th argument: additional information
% This parameter is NOT checked at the pdfLaTeX compilation.
%
% All parameters are mandatory and cannot be empty. Do NOT specify
% Instrument Configurations for alternative runs.

% Examples (to be commented or deleted)

% \INSconfig{A}{FORS2}{Detector}{E2V}
% \INSconfig{A}{FORS2}{IMG}{ESO filters: provide list HERE}
% \INSconfig{B}{FORS2}{Detector}{MIT}
% \INSconfig{B}{FORS2}{IMG}{ESO filters: provide list HERE}
% \INSconfig{C}{HARPS}{spectro-polarimetry}{linear}
% \INSconfig{D}{FORS2}{Detector}{MIT}
% \INSconfig{D}{FORS2}{IMG}{ESO filters: provide list HERE}

%
% Real list of instrument configurations

%%%%%%%%%%%%%%%%%%%%%%%%%%%%%%%%%%%%%%%%%%%%%%%%%%%%%%%%%%%%%%%%%%%%%%%%%
% Paranal
%
%-----------------------------------------------------------------------
%---- FORS2 at the VLT-UT1 (ANTU) --------------------------------------
%-----------------------------------------------------------------------
%If you require the E2V (Blue) detector uncomment the following line
%\INSconfig{}{FORS2}{Detector}{E2V}
%
%If you require the MIT (RED) detector uncomment the following line
%\INSconfig{}{FORS2}{Detector}{MIT}
%
% If you require the High-Resolution  collimator uncomment the following line
%\INSconfig{}{FORS2}{collimator}{HR}
%
% Uncomment the line(s) corresponding to the imaging mode(s) you require and
% provide the list of filters needed  for your observations:
%
%\INSconfig{}{FORS2}{PRE-IMG}{ESO filters: provide list HERE}
%\INSconfig{}{FORS2}{IMG}{ESO filters: provide list HERE}
%\INSconfig{}{FORS2}{IMG}{User's own filters (to be described in text)}
%\INSconfig{}{FORS2}{IPOL}{ESO filters: provide list HERE}
%\INSconfig{}{FORS2}{IPOL}{User's own filters (to be described in text)}
%\INSconfig{}{FORS2}{HIT-MS}{Provide list of grisms HERE}
%
%
% Uncomment the line(s) corresponding to the spectroscopic mode(s) you require and
% provide the list of grism+filter combination needed  for your observations:
%
%\INSconfig{}{FORS2}{LSS}{Provide list of grism+filter combinations HERE}
%\INSconfig{}{FORS2}{MOS}{Provide list of grism+filter combinations HERE}
%\INSconfig{}{FORS2}{PMOS}{Provide list of grism+filter combinations HERE}
%\INSconfig{}{FORS2}{MXU}{Provide list of grism+filter combinations HERE}
%\INSconfig{}{FORS2}{HITI}{ESO filters: provide list HERE}
%\INSconfig{}{FORS2}{HIT-OS}{Provide list of grisms HERE}
%
% Uncomment the following line for Rapid Response Mode observations
%
%\INSconfig{}{FORS2}{RRM}{yes}
%
% Uncomment the following line for use of the Virtual Image Slicer
%\INSconfig{}{FORS2}{Virtual Image Slicer}{VM only}
%-----------------------------------------------------------------------
%---- KMOS at the VLT-UT1 (ANTU) ---------------------------------------
%-----------------------------------------------------------------------
%
%\INSconfig{}{KMOS}{IFU}{provide list of settings (IZ, YJ, H, K, HK) here} 
%
%-----------------------------------------------------------------------
%---- FLAMES at the VLT-UT2 (KUEYEN) -----------------------------------
%-----------------------------------------------------------------------
%\INSconfig{}{FLAMES}{UVES}{Specify the UVES setup below}
%\INSconfig{}{FLAMES}{GIRAFFE-MEDUSA}{Specify the GIRAFFE setup below}
%\INSconfig{}{FLAMES}{GIRAFFE-IFU}{Specify the GIRAFFE setup below}
%\INSconfig{}{FLAMES}{GIRAFFE-ARGUS}{Specify the GIRAFFE setup below}
%\INSconfig{}{FLAMES}{Combined: UVES + GIRAFFE-MEDUSA}{Specify the UVES and
%GIRAFFE setups below}
%\INSconfig{}{FLAMES}{Combined: UVES + GIRAFFE-IFU}{Specify the UVES and
%GIRAFFE setups below}
%\INSconfig{}{FLAMES}{Combined: UVES + GIRAFFE-ARGUS}{Specify the UVES and
%GIRAFFe setups below}
%
%
% If you have selected UVES, either standalone or in combined mode,
% please specify the UVES standard setup(s) to be used
%\INSconfig{}{FLAMES}{UVES}{standard setup Red 520}
%\INSconfig{}{FLAMES}{UVES}{standard setup Red 580}
%\INSconfig{}{FLAMES}{UVES}{standard setup Red 580 + simultaneous calibration}
%\INSconfig{}{FLAMES}{UVES}{standard setup Red 860}
%
%\INSconfig{}{FLAMES}{GIRAFFE}{fast readout mode 625kHz VM only}
%\INSconfig{}{FLAMES}{GIRAFFE}{slow readout mode 50kHz VM only}
%
% If you have selected GIRAFFE, either standalone or in combined mode
% please specify the GIRAFFE standard setups(s) to be used
%\INSconfig{}{FLAMES}{GIRAFFE}{standard setup HR01 379.0}
%\INSconfig{}{FLAMES}{GIRAFFE}{standard setup HR02 395.8}
%\INSconfig{}{FLAMES}{GIRAFFE}{standard setup HR03 412.4}
%\INSconfig{}{FLAMES}{GIRAFFE}{standard setup HR04 429.7}
%\INSconfig{}{FLAMES}{GIRAFFE}{standard setup HR05 447.1 A}
%\INSconfig{}{FLAMES}{GIRAFFE}{standard setup HR05 447.1 B}
%\INSconfig{}{FLAMES}{GIRAFFE}{standard setup HR06 465.6}
%\INSconfig{}{FLAMES}{GIRAFFE}{standard setup HR07 484.5 A}
%\INSconfig{}{FLAMES}{GIRAFFE}{standard setup HR07 484.5 B}
%\INSconfig{}{FLAMES}{GIRAFFE}{standard setup HR08 504.8}
%\INSconfig{}{FLAMES}{GIRAFFE}{standard setup HR09 525.8 A}
%\INSconfig{}{FLAMES}{GIRAFFE}{standard setup HR09 525.8 B}
%\INSconfig{}{FLAMES}{GIRAFFE}{standard setup HR10 548.8}
%\INSconfig{}{FLAMES}{GIRAFFE}{standard setup HR11 572.8}
%\INSconfig{}{FLAMES}{GIRAFFE}{standard setup HR12 599.3}
%\INSconfig{}{FLAMES}{GIRAFFE}{standard setup HR13 627.3}
%\INSconfig{}{FLAMES}{GIRAFFE}{standard setup HR14 651.5 A}
%\INSconfig{}{FLAMES}{GIRAFFE}{standard setup HR14 651.5 B}
%\INSconfig{}{FLAMES}{GIRAFFE}{standard setup HR15 665.0}
%\INSconfig{}{FLAMES}{GIRAFFE}{standard setup HR15 679.7}
%\INSconfig{}{FLAMES}{GIRAFFE}{standard setup HR16 710.5}
%\INSconfig{}{FLAMES}{GIRAFFE}{standard setup HR17 737.0 A}
%\INSconfig{}{FLAMES}{GIRAFFE}{standard setup HR17 737.0 B}
%\INSconfig{}{FLAMES}{GIRAFFE}{standard setup HR18 769.1}
%\INSconfig{}{FLAMES}{GIRAFFE}{standard setup HR19 805.3 A}
%\INSconfig{}{FLAMES}{GIRAFFE}{standard setup HR19 805.3 B}
%\INSconfig{}{FLAMES}{GIRAFFE}{standard setup HR20 836.6 A}
%\INSconfig{}{FLAMES}{GIRAFFE}{standard setup HR20 836.6 B}
%\INSconfig{}{FLAMES}{GIRAFFE}{standard setup HR21 875.7}
%\INSconfig{}{FLAMES}{GIRAFFE}{standard setup HR22 920.5 A}
%\INSconfig{}{FLAMES}{GIRAFFE}{standard setup HR22 920.5 B}
%\INSconfig{}{FLAMES}{GIRAFFE}{standard setup LR01 385.7}
%\INSconfig{}{FLAMES}{GIRAFFE}{standard setup LR02 427.2}
%\INSconfig{}{FLAMES}{GIRAFFE}{standard setup LR03 479.7}
%\INSconfig{}{FLAMES}{GIRAFFE}{standard setup LR04 543.1}
%\INSconfig{}{FLAMES}{GIRAFFE}{standard setup LR05 614.2}
%\INSconfig{}{FLAMES}{GIRAFFE}{standard setup LR06 682.2}
%\INSconfig{}{FLAMES}{GIRAFFE}{standard setup LR07 773.4}
%\INSconfig{}{FLAMES}{GIRAFFE}{standard setup LR08 881.7}
%
%\INSconfig{}{FLAMES}{GIRAFFE}{fast readout mode 625kHz VM only}
%
%-----------------------------------------------------------------------
%---- X-SHOOTER at the VLT-UT2 (KUEYEN)
%-----------------------------------------------------------------------
%
%\INSconfig{}{XSHOOTER}{300-2500nm}{SLT}
%\INSconfig{}{XSHOOTER}{300-2500nm}{IFU}
%
% Slits (SLT only):
%
%UVB arm, available slits in arcsec: 0.5, 0.8, 1.0, 1.3, 1.6, 5.0
%VIS arm, available slits in arcsec: 0.4, 0.7, 0.9, 1.2, 1.5, 5.0 
%NIR arm, available slits in arcsec: 0.4, 0.6, 0.6JH, 0.9, 0.9JH, 1.2, 5.0
%  The 0.6JH and 0.9JH include a stray light K-band blocking filter
%  that allow sky limited studies in J and H bands.
%
%The slits for IFU  are fixed and do not need to be mentioned here.
%
% Replace SLIT-UVB, SLIT-VIS, SLIT-NIR with the choice of the slits:
%\INSconfig{}{XSHOOTER}{SLT}{SLIT-UVB,SLIT-VIS,SLIT-NIR}
%
% Detector readout mode:
%
% UVB and VIS arms: available readout modes and binning:
% 100k-1x1, 100k-1x2, 100k-2x2, 400k-1x1, 400k-1x2, 400k-2x2
% The NIR readout mode is fixed  to NDR.
%
%\INSconfig{}{XSHOOTER}{IFU}{readout UVB,readout VIS,readout NIR}
%\INSconfig{}{XSHOOTER}{SLT}{readout UVB,readout VIS,readout NIR}
%
% Imaging mode 
% replace 'list of filters' by the actual filters you wish to use among:
% U, B, V, R, I, Uprime, Gprime, Rprime, Iprime, Zprime
% Please note that the imaging mode can only be used in combination with slit or IFU observations
%\INSconfig{}{XSHOOTER}{IMG}{list of filters}
%
%\INSconfig{}{XSHOOTER}{RRM}{yes}
%
%-----------------------------------------------------------------------
%---- UVES at the VLT-UT2 (KUEYEN) -------------------------------------
%-----------------------------------------------------------------------
%
%\INSconfig{}{UVES}{BLUE}{Standard setting: 346}
%\INSconfig{}{UVES}{BLUE}{Standard setting: 437}
%\INSconfig{}{UVES}{BLUE}{Non-std setting: provide central wavelength  HERE}
%
%\INSconfig{}{UVES}{RED}{Standard setting: 520}
%\INSconfig{}{UVES}{RED}{Standard setting: 580}
%\INSconfig{}{UVES}{RED}{Standard setting: 600}
%\INSconfig{}{UVES}{RED}{Iodine cell standard setting: 600}
%\INSconfig{}{UVES}{RED}{Standard setting: 860}
%\INSconfig{}{UVES}{RED}{Non-std setting: provide central wavelength HERE}
%
%\INSconfig{}{UVES}{DIC-1}{Standard setting: 346+580}
%\INSconfig{}{UVES}{DIC-1}{Standard setting: 390+564}
%\INSconfig{}{UVES}{DIC-1}{Standard setting: 346+564}
%\INSconfig{}{UVES}{DIC-1}{Standard setting: 390+580}
%\INSconfig{}{UVES}{DIC-1}{Non-std setting: provide central wavelength HERE}
%
%\INSconfig{}{UVES}{DIC-2}{Standard setting: 437+860}
%\INSconfig{}{UVES}{DIC-2}{Standard setting: 346+860}
%\INSconfig{}{UVES}{DIC-2}{Standard setting: 390+860}
%
%\INSconfig{}{UVES}{DIC-2}{Standard setting: 437+760}
%\INSconfig{}{UVES}{DIC-2}{Standard setting: 346+760}
%\INSconfig{}{UVES}{DIC-2}{Standard setting: 390+760}
%\INSconfig{}{UVES}{DIC-2}{Non-std setting: provide central wavelength HERE}
%
%\INSconfig{}{UVES}{Field Derotation}{yes}
%\INSconfig{}{UVES}{Image slicer-1}{yes}
%\INSconfig{}{UVES}{Image slicer-2}{yes}
%\INSconfig{}{UVES}{Image slicer-3}{yes}
%\INSconfig{}{UVES}{Iodine cell}{yes}
%\INSconfig{}{UVES}{Longslit Filters}{Provide list of filters HERE}
%
%\INSconfig{}{UVES}{RRM}{yes}
%
%-----------------------------------------------------------------------
%---- SPHERE at the VLT-UT3 (MELIPAL) -----------------------------------
%-----------------------------------------------------------------------
%
%
% Pupil or field tracking?
% Mode choices: IRDIS-CI, IRDIS-DBI, 
%               IRDIFS, IRDIFS-EXT, 
%               ZIMPOL-I
%               (Not relevant for IRDIS-DPI, IRDIS-LSS, ZIMPOL-P1 or ZIMPOL-P2)
%--------------------
% IRDIFS: 
% Coronagraph combination choices:
%   IRDIFS:     None, N-ALC-YJH-S, N-ALC-YJH-L, N-CLC-SW-L
%   IRDIFS-EXT: None, N-ALC-YJH-S, N-ALC-YJH-L, N-ALC-Ks
% Filter choices for IRDIS in IRDIFS mode
%   IRDIFS:     DB-H23, DB-ND23, DB-H34, BB-H
%   IRDIFS-EXT: DB-K12, BB-Ks
%---------------------
% IRDIS imaging (alone):
% Coronagraph combination choices for IRDIS imaging modes (see UM for details)
%   IRDIS-CI, IRDIS-DPI:  
%              None, N-ALC-Y, N-ALC-YJ-S, N-ALC-YJ-L, N-ALC-YJH-S, 
%                    N-ALC-YJH-L, N-ALC-Ks
%   IRDIS-DBI: None, N-ALC-Y, N-ALC-YJ-S, N-ALC-YJ-L, N-ALC-YJH-S, 
%                    N-ALC-YJH-L, N-ALC-Ks
% Filter choices:
%   IRDIS-CI, IRDIS-DPI: 
%              BB-Y, BB-J, BB-H, BB-Ks, NB-Hel, NB-CntJ, NB-CntH,
%              NB-CntK1, NB-BrG, NB-CntK2, NB-PaB, NB-FeII, NB-H2, NB-CO
%   IRDIS-DBI: DB-Y23, DB-J23, DB-H23, DB-NDH23,  DB-H34, DB-K12 
%---------------------
% IRDIS spectroscopy:
% Coronagraphic slit/grism combinations for IRDIS-LSS:
%   IRDIS-LSS: N-S-LR-WL, N-S-MR-WL
%---------------------
% ZIMPOL imaging: 
% Coronagraph choices:
%   ZIMPOL-I: None, V-CLC-M-WF, V-CLC-M-NF, V-CLC-L-WF, V-CLC-XL-WF
% Filter choices:
%   ZIMPOL-I: RI, R-PRIM, I-PRIM, V, V-S, V-L, N-R, 730-NB, N-I, I-L,
%             KI,  TiO-717, CH4-727, Cnt748, Cnt820, HeI, OI-630,
%             CntHa, B-Ha, N-Ha, Ha-NB
%--------------------
% ZIMPOL polarimetry:
% Coronagraph choices:
%    ZIMPOL-P1: None, V-CLC-S-WF, V-CLC-M-WF, V-CLC-L-WF, V-CLC-XL-WF, V-CLC-MT-WF
%    ZIMPOL-P2: None, V-CLC-S-WF, V-CLC-M-WF, V-CLC-L-WF, V-CLC-XL-WF, V-CLC-MT-WF
% Filter choices:
%    ZIMPOL-P1: RI, R-PRIM, I-PRIM, V, N-R, N-I, KI, TiO-717, 
%               CH4-727, Cnt748, Cnt820, CntHa, N-Ha, B-Ha     
%    ZIMPOL-P2: RI, R-PRIM, I-PRIM, V, N-R, N-I, KI, TiO-717, 
%               CH4-727, Cnt748, Cnt820, CntHa, N-Ha, B-Ha 
% Readout mode choice for ZIMPOL
%    ZIMPOL-P1: FastPol, SlowPol
%    ZIMPOL-P2: FastPol, SlowPol
%-------------------
%
% One entry per mode. Repeat the entry for each mode.
%
%\INSconfig{}{SPHERE}{Pupil}{mode}
%\INSconfig{}{SPHERE}{Field}{mode}
%
% One entry per combination. Repeat the entry for each combination.
%
%\INSconfig{}{SPHERE}{IRDIFS}{Coronagraph/filter combination for IRDIFS}
%\INSconfig{}{SPHERE}{IRDIFS-EXT}{Coronagraph/filter combination for IRDIFS-EXT}
%
%\INSconfig{}{SPHERE}{IRDIS-CI}{Coronagraph/filter combination for IRDIS-CI}
%\INSconfig{}{SPHERE}{IRDIS-DBI}{Coronagraph/filter combination for IRDIS-DBI}
%\INSconfig{}{SPHERE}{IRDIS-DPI}{Coronagraph/filter combination for IRDIS-DPI}
%\INSconfig{}{SPHERE}{IRDIS-LSS}{Coronagraphic slit/grism combination for IRDIS-LSS}
%
%\INSconfig{}{SPHERE}{ZIMPOL-I}{Coronagraph/filter combination for ZIMPOL-I}
%
%\INSconfig{}{SPHERE}{ZIMPOL-P1}{Coronagraph/filter/readout mode for ZIMPOL-P1}
%\INSconfig{}{SPHERE}{ZIMPOL-P2}{Coronagraph/filter/readout mode for ZIMPOL-P2}
%
%
%-----------------------------------------------------------------------
%---- SINFONI at the VLT-UT4 (YEPUN) -----------------------------------
%-----------------------------------------------------------------------
%

%\INSconfig{}{SINFONI}{PRE-IMG}{provide list of setting(s) (J,H,K,H+K)}
%
%\INSconfig{}{SINFONI}{IFS 250mas/pix no-AO}{provide list of setting(s) (J,H,K,H+K) HERE}
%\INSconfig{}{SINFONI}{IFS 100mas/pix no-AO}{provide list of setting(s) (J,H,K,H+K) HERE}
%
% If you plan to use a NGS, please specify the NGS name and magnitude (Rmag preferred,
% otherwise Vmag) in target list.
%\INSconfig{}{SINFONI}{IFS 250mas/pix NGS}{provide list of setting(s) (J,H,K,H+K) HERE}
%\INSconfig{}{SINFONI}{IFS 100mas/pix NGS}{provide list of setting(s) (J,H,K,H+K) HERE}
%\INSconfig{}{SINFONI}{IFS 25mas/pix NGS}{provide list of setting(s) (J,H,K,H+K) HERE}
%
% If you plan to use the LGS, please specify the TTS name and magnitude (Rmag preferred,
% otherwise Vmag) in target list.
%\INSconfig{}{SINFONI}{IFS 250mas/pix LGS}{provide list of setting(s) (J,H,K,H+K) HERE}
%\INSconfig{}{SINFONI}{IFS 100mas/pix LGS}{provide list of setting(s) (J,H,K,H+K) HERE}
%\INSconfig{}{SINFONI}{IFS 25mas/pix LGS}{provide list of setting(s) (J,H,K,H+K) HERE}
%
% If you plan to use the LGS without a TTS (seeing enhancer mode) then
% please leave the TTS name blank in the target list.
%\INSconfig{}{SINFONI}{IFS 250mas/pix LGS-noTTS}{provide list of setting(s) (J,H,K,H+K) HERE}
%\INSconfig{}{SINFONI}{IFS 100mas/pix LGS-noTTS}{provide list of setting(s) (J,H,K,H+K) HERE}
%\INSconfig{}{SINFONI}{IFS 25mas/pix LGS-noTTS}{provide list of setting(s) (J,H,K,H+K) HERE}
%
% Select if you have special calibrations
%\INSconfig{}{SINFONI}{Special Cal}{-}
%
% Select if you need pupil tracking mode
%\INSconfig{}{SINFONI}{Pupil Track}{-}
%
% Select for RRM
%\INSconfig{}{SINFONI}{RRM}{yes}
%
%-----------------------------------------------------------------------
%---- MUSE at the VLT-UT4 (YEPUN) -----------------------------------
%-----------------------------------------------------------------------
%
\INSconfig{A}{MUSE}{WFM-NOAO-N}{-}
\INSconfig{B}{MUSE}{WFM-NOAO-N}{-}
\INSconfig{C}{MUSE}{WFM-NOAO-N}{-}
\INSconfig{D}{MUSE}{WFM-NOAO-N}{-}
%\INSconfig{}{MUSE}{WFM-NOAO-E}{-}
%
% Uncomment the following line for Rapid Response Mode observations
%\INSconfig{}{MUSE}{RRM}{yes}
%
%%%%%%%%%%%%%%%%%%%%%%%%%%%%%%%%%%%%%%%%%%%%%%%%%%%%%%%%%%%%%%%%%%%%%%%%
%-----------------------------------------------------------------------
%---- PIONIER ----------------------------------------------------------
%-----------------------------------------------------------------------
%
%
%\INSconfig{}{PIONIER}{GRISM}{1.65}
%\INSconfig{}{PIONIER}{FREE}{1.65}
%
%%%%%%%%%%%%%%%%%%%%%%%%%%%%%%%%%%%%%%%%%%%%%%%%%%%%%%%%%%%%%%%%%%%%%%%%
%-----------------------------------------------------------------------
%---- OMEGACAM at VST --------------------------------------------------
% This instrument is only available for GTO, Chilean and filler programmes.
%-----------------------------------------------------------------------
%
%\INSconfig{}{OMEGACAM}{IMG}{provide list of filters here}
%
%%%%%%%%%%%%%%%%%%%%%%%%%%%%%%%%%%%%%%%%%%%%%%%%%%%%%%%%%%%%%%%%%%%%%%%%
% La Silla
%-----------------------------------------------------------------------
%---- EFOSC2 (or SOFOSC) at the NTT ------------------------------------
%-----------------------------------------------------------------------
%
%\INSconfig{}{EFOSC2}{PRE-IMG}{EFOSC2 filters: provide list here}
%\INSconfig{}{EFOSC2}{Imaging-filters}{EFOSC2 filters:  provide list here}
%\INSconfig{}{EFOSC2}{Imaging-filters}{ESO non EFOSC filters: provide ESOfilt No}
%\INSconfig{}{EFOSC2}{Imaging-filters}{User's own filters (to be described in text)}
%\INSconfig{}{EFOSC2}{Spectro-long-slit}{Grism\#1:320-1090}
%\INSconfig{}{EFOSC2}{Spectro-long-slit}{Grism\#2:510-1100}
%\INSconfig{}{EFOSC2}{Spectro-long-slit}{Grism\#3:305-610}
%\INSconfig{}{EFOSC2}{Spectro-long-slit}{Grism\#4:409-752}
%\INSconfig{}{EFOSC2}{Spectro-long-slit}{Grism\#5:520-935}
%\INSconfig{}{EFOSC2}{Spectro-long-slit}{Grism\#6:386-807}
%\INSconfig{}{EFOSC2}{Spectro-long-slit}{Grism\#7:327-524}
%\INSconfig{}{EFOSC2}{Spectro-long-slit}{Grism\#8:432-636}
%\INSconfig{}{EFOSC2}{Spectro-long-slit}{Grism\#11:338-752}
%\INSconfig{}{EFOSC2}{Spectro-long-slit}{Grism\#13:369-932}
%\INSconfig{}{EFOSC2}{Spectro-long-slit}{Grism\#14:310-509}
%\INSconfig{}{EFOSC2}{Spectro-long-slit}{Grism\#16:602-1032}
%\INSconfig{}{EFOSC2}{Spectro-long-slit}{Grism\#17:689-876}
%\INSconfig{}{EFOSC2}{Spectro-long-slit}{Grism\#18:470-677}
%\INSconfig{}{EFOSC2}{Spectro-long-slit}{Grism\#19:440-510}
%\INSconfig{}{EFOSC2}{Spectro-long-slit}{Grism\#20:605:715}
%\INSconfig{}{EFOSC2}{Spectro-long-slit}{Aperture: 0.5'', ... ,10.0''}
%
%\INSconfig{}{EFOSC2}{Spectro-long-slit}{Aperture: Shiftable}
%\INSconfig{}{EFOSC2}{Spectro-MOS}{PunchHead=0.95''}
%\INSconfig{}{EFOSC2}{Spectro-MOS}{PunchHead=1.12''}
%\INSconfig{}{EFOSC2}{Spectro-MOS}{PunchHead=1.45''}
%\INSconfig{}{EFOSC2}{Polarimetry}{$\lambda / 2$ retarder plate}
%\INSconfig{}{EFOSC2}{Polarimetry}{$\lambda / 4$ retarder plate}
%\INSconfig{}{EFOSC2}{Coronograph}{yes}
%
%
%-----------------------------------------------------------------------
%---- SOFI (or SOFOSC) at the NTT --------------------------------------------------
%-----------------------------------------------------------------------
%
%\INSconfig{}{SOFI}{PRE-IMG-LargeField}{Provide list of filters HERE}
%\INSconfig{}{SOFI}{Imaging-LargeField}{Provide list of filters HERE}
%\INSconfig{}{SOFI}{Burst}{Provide list of filters HERE}
%\INSconfig{}{SOFI}{FastPhot}{Provide list of filters HERE}
%\INSconfig{}{SOFI}{Polarimetry}{Provide list of filters HERE}
%\INSconfig{}{SOFI}{Spectroscopy-long-slit}{Blue Grism, Provide list of slits HERE}
%\INSconfig{}{SOFI}{Spectroscopy-long-slit}{Red Grism, Provide list of slits HERE}
%\INSconfig{}{SOFI}{Spectroscopy-high-res}{H, Provide list of slits HERE}
%\INSconfig{}{SOFI}{Spectroscopy-high-res}{K, Provide list of slits HERE}
%
%
%-----------------------------------------------------------------------
%---- HARPS at the 3.6 -------------------------------------------------
%-----------------------------------------------------------------------
%
%\INSconfig{}{HARPS}{spectro-Thosimult}{HARPS}
%\INSconfig{}{HARPS}{WAVE}{HARPS}
%\INSconfig{}{HARPS}{spectro-ObjA(B)}{HARPS}
%\INSconfig{}{HARPS}{spectro-ObjA(B)}{EGGS}
%\INSconfig{}{HARPS}{spectro-polarimetry}{linear}
%\INSconfig{}{HARPS}{spectro-polarimetry}{circular}
%
%
%%%%%%%%%%%%%%%%%%%%%%%%%%%%%%%%%%%%%%%%%%%%%%%%%%%%%%%%%%%%%%%%%%%%%%%%
% Chajnantor
%-----------------------------------------------------------------------
%---- SHFI at APEX ----------------------------------------------
%-----------------------------------------------------------------------
%
%\INSconfig{}{SHFI}{APEX-1}{Please enter Central Frequency 211 to 275 GHz}
%\INSconfig{}{SHFI}{APEX-2}{Please enter Central Frequency 275 to 370 GHz}
%\INSconfig{}{SHFI}{APEX-3}{Please enter Central Frequency 385 to 500 GHz} 
%
%-----------------------------------------------------------------------
%---- LABOCA at APEX ----------------------------------------------
%-----------------------------------------------------------------------
%
%\INSconfig{}{LABOCA}{IMG}{-}
%
%-----------------------------------------------------------------------
%-----------------------------------------------------------------------
%---- SEPIA at APEX ----------------------------------------------
%-----------------------------------------------------------------------
%
%\INSconfig{}{SEPIA}{Band-5}{Please enter Central Frequency 159 to 211 GHz}
%\INSconfig{}{SEPIA}{Band-9}{Please enter Central Frequency 602 to 720 GHz}
%


%%%%%%%%%%%%%%%%%%%%%%%%%%%%%%%%%%%%%%%%%%%%%%%%%%%%%%%%%%%%%%%%%%%%%%%%
%%%%% Interferometry PAGE %%%%%%%%%%%%%%%%%%%%%%%%%%%%%%%%%%%%%%%%%%%%%%
%%%%%%%%%%%%%%%%%%%%%%%%%%%%%%%%%%%%%%%%%%%%%%%%%%%%%%%%%%%%%%%%%%%%%%%%
%
% The \VLTITarget macro is only needed when requesting
% Interferometry, in which case it is MANDATORY to uncomment it and
% fill in the information. It takes the following parameters:
%
% 1st argument: run ID
% Valid values: run IDs specified in BOX 3
%
% 2nd argument: target name
% This parameter is NOT checked at the pdfLaTeX compilation.
%
% 3rd argument: visual magnitude
% Values with up to decimal places are allowed here.
% This parameter is NOT checked at the pdfLaTeX compilation.
%
% 4th argument: magnitude at wavelength of observation
% Values with up to decimal places are allowed here.
% This parameter is NOT checked at the pdfLaTeX compilation.
%
% 5th argument: wavelength of observation (in microns)
% Values with up to decimal places are allowed here.
% This parameter is NOT checked at the pdfLaTeX compilation.
%
% 6th argument: size at wavelength of observation (in mas)
% This parameter is NOT checked at the pdfLaTeX compilation.
%
%
% 7th argument: baseline
% UT observations are scheduled in terms of 3-telescope
% baselines for AMBER and 4-telescope baselines for PIONIER.
% For UT observations please specify one of the available
% AMBER or PIONIER baselines.
%
% AT observations are scheduled in terms of 4-telescope
% configurations (quadruplets). For these observations, the
% time can be split among the different baselines; the exact
% baselines will be specified at Phase 2.
% For AT observations, please specify only one of the 3
% available AT quadruplets at this stage.
%
%
% 8th parameter: Range of visibilities for the specified configuration.
% Please specify the maximum and minimum visibility values
% corresponding to the chosen configuration at hour angle 0
% separated by "/".
% This parameter is NOT checked at the pdfLaTeX compilation. 
%
%
% 9th parameter: correlated magnitude
% (for the visibility values specified in the 8th parameter)
% This parameter is NOT checked at the pdfLaTeX compilation.
%
%
% 10th parameter: time on target in hours
% Values with up to decimal places are allowed here.
% This parameter is NOT checked at the pdfLaTeX compilation.
%
% The available baselines for Period 100 are shown below.
% For AT observations, the time can be split at Phase 2 among the
% different offered baselines of the chosen quadruplet. 
% Please see the Call for Proposals for more information.
%
% 
% AMBER
% A0-B2-C1-D0
% A0-G1-J2-J3
% D0-G2-J3-K0
% UT1-UT2-UT3
% UT1-UT2-UT4
% UT1-UT3-UT4
% UT2-UT3-UT4
% 
% GRAVITY
% A0-B2-C1-D0
% A0-G1-J2-J3
% D0-G2-J3-K0
% UT1-UT2-UT3-UT4
% 
% PIONIER
% A0-B2-C1-D0
% A0-G1-J2-J3
% D0-G2-J3-K0
% UT1-UT2-UT3-UT4
% 

%\VLTITarget{E}{Alpha Ori}{-1.4}{-1.4}{10.6}{6}{UT1-UT2-UT3}{0.60/0.10}{-0.2/4.0}{2} 
%\VLTITarget{F}{Alpha Ori}{-1.4}{-1.4}{10.6}{6}{D0-K0-G2-J3}{0.80/0.40}{-0.9/-0.2}{1} 

% You can specify here a note applying to all or some of your VLTI
% targets.  You should take advantage of this note to indicate
% suitable alternative baselines for your observations.
% This macro is NOT checked at the pdfLaTeX compilation.

%\VLTITargetNotes{Note about the VLTI targets, e.g., Run E can also be carried out using UT1-UT3-UT4.}



%%%%%%%%%%%%%%%%%%%%%%%%%%%%%%%%%%%%%%%%%%%%%%%%%%%%%%%%%%%%%%%%%%%%%%%%
%%%%% VISITOR SPECIAL INSTRUMENT PAGE %%%%%%%%%%%%%%%%%%%%%%%%%%%%%%%%%%
%%%%%%%%%%%%%%%%%%%%%%%%%%%%%%%%%%%%%%%%%%%%%%%%%%%%%%%%%%%%%%%%%%%%%%%%
%
% The following commands are only needed when bringing a Visitor
% Special Instrument, in which case it is MANDATORY to uncomment them
% and provide all the required information.
%
%\Desc{}   %Description of the instrument and its operation
%\Comm{}   %On which telescope(s) has instrument been commissioned/used
%\WV{}     %Total weight and value of equipment to be shipped
%\Wfocus{} %Weight at the focus (including ancillary equipment)
%\Interf{} %Compatibility of attachment interface with required focus
%\Focal{}  %Back focal distance value
%\Acqu{}   %Acquisition, focusing, and guiding procedure
%\Softw{}  %Compatibility with ESO software standards (data handling)
%\Suppl{}  %Estimate of services expected from ESO (in person days)

%%%%%%%%%%%%%%%%%%%%%%%%%%%%%%%%%%%%%%%%%%%%%%%%%%%%%%%%%%%%%%%%%%%%%%%%
%%%%% THE END %%%%%%%%%%%%%%%%%%%%%%%%%%%%%%%%%%%%%%%%%%%%%%%%%%%%%%%%%%
%%%%%%%%%%%%%%%%%%%%%%%%%%%%%%%%%%%%%%%%%%%%%%%%%%%%%%%%%%%%%%%%%%%%%%%%
\MakeProposal
\end{document}


