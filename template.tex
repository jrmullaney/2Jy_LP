%%%%%%%%%%%%%%%%%%%%%%%%%%%%%%%%%%%%%%%%%%%%%%%%%%%%%%%%%%%%%%%%%%%%%%
%
%.IDENTIFICATION $Id: template.tex.src,v 1.41 2008/01/25 10:47:12 fsogni Exp $
%.LANGUAGE       TeX, LaTeX
%.ENVIRONMENT    ESOFORM
%.PURPOSE        Template application form for ESO Observing time.
%.AUTHOR         The Esoform Package is maintained by the Observing
%                Programmes Office (OPO) while the background software
%                is provided by the User Support System (USS) Department.
%
%-----------------------------------------------------------------------
%
%
%                   ESO LA SILLA PARANAL OBSERVATORY
%                   --------------------------------
%                   NORMAL PROGRAMME PHASE 1 TEMPLATE
%                   ---------------------------------
%
%
%
%          PLEASE CHECK THE ESOFORM USERS' MANUAL FOR DETAILED 
%              INFORMATION AND DESCRIPTIONS OF THE MACROS. 
%     (see the file usersmanual.tex provided in the ESOFORM package) 
%
%
%        ====>>>> TO BE SUBMITTED THROUGH WEB UPLOAD  <<<<====
%               (see the Call for Proposals for details)
%
%%%%%%%%%%%%%%%%%%%%%%%%%%%%%%%%%%%%%%%%%%%%%%%%%%%%%%%%%%%%%%%%%%%%%%

%%%%%%%%%%%%%%%%%%%%%%%%%%%%%%%%%%%%%%%%%%%%%%%%%%%%%%%%%%%%%%%%%%%%%%
%
%                      I M P O R T A N T    N O T E
%                      ----------------------------
%
% By submitting this proposal, the Principal Investigator takes full
% responsibility for the content of the proposal, in particular with
% regard to the names of CoI's and the agreement to act in accordance
% with the ESO policy and regulations, should observing time be
% granted.
%
%%%%%%%%%%%%%%%%%%%%%%%%%%%%%%%%%%%%%%%%%%%%%%%%%%%%%%%%%%%%%%%%%%%%%% 

%
%    - LaTeX *is* sensitive towards upper and lower case letters.
%    - Everything after a `%' character is taken as comments.
%    - DO NOT CHANGE ANY OF THE MACRO NAMES (words beginning with `\')
%    - DO NOT INSERT ANY TEXT OUTSIDE THE PROVIDED MACROS
%

%
%    - All parameters are checked at the verification or submission.
%    - Some parameters are also checked during the pdfLaTeX
%      compilation.  If this is not the case, this is indicated by the
%      phrase
%      "This parameter is NOT checked at the pdfLaTeX compilation."
%

\documentclass{esoform}

% The list of LaTeX definitions of commonly used astronomical symbols
% is already included in the style file common2e.sty (see Table 1 in
% the Users' Manual).  If you have your own macros or definitions,
% please insert them here, between the \documentclass{esoform}
% and the \begin{document} commands.
%
%     PLEASE USE NEITHER YOUR OWN MACROS NOR ANY TEX/LATEX MACROS  
%       IN THE \Title, \Abstract, \PI, \CoI, and \Target MACROS.
%
% WARNING: IT IS THE RESPONSIBILITY OF THE APPLICANTS TO STAY WITHIN THE
% CURRENT BOX LIMITS AND ELIMINATE POTENTIAL OVERFILL/OVERWRITE PROBLEMS 

\begin{document}

%%%%%%%%%%%%%%%%%%%%%%%%%%%%%%%%%%%%%%%%%%%%%%%%%%%%%%%%%%%%%%%%%%%%%%%%
%%%%% CONTENTS OF THE FIRST PAGE %%%%%%%%%%%%%%%%%%%%%%%%%%%%%%%%%%%%%%%
%%%%%%%%%%%%%%%%%%%%%%%%%%%%%%%%%%%%%%%%%%%%%%%%%%%%%%%%%%%%%%%%%%%%%%%%
%
%---- BOX 1 ------------------------------------------------------------
%
% You should use this template for period 99A applications ONLY.
%
% DO NOT EDIT THE MACRO BELOW. 

\Cycle{101A}

% Type below, within the curly braces {}, the title of your observing
% programme (up to two lines).
% This parameter is NOT checked at the pdfLaTeX compilation.
%
% DO NOT USE ANY TEX/LATEX MACROS IN THE TITLE

\Title{This Is The Proposal Title This Is The Proposal Title}  

% Type below the numeric code corresponding to the subcategory of your
% programme.

\SubCategoryCode{A9}   

% Please specify the type of programme you are submitting. 
% Valid values: NORMAL, GTO, TOO, CALIBRATION, MONITORING
% If you specify TOO, you will also need to fill a ToO page below.
% If you specify CALIBRATION, then the SubCategory Code MUST be set to L0

% If your programme requires more than 100 hours the Large Programme
% template (templatelarge.tex) must be used.


\ProgrammeType{NORMAL}

% For GTO proposals only: uncomment the following and fill out the GTO
% programme code (as communicated to the respective GTO coordinator).

%\GTOcontract{INS-consortium}		

% For TOO proposals only: uncomment the following if you apply for
% Rapid Response Mode observations.
 
%\ObservationInRRM{}

% Uncomment the following macro if this proposal is applying for time
% under the VLT-XMM agreement (only available for odd periods).

%\ObservationWithXMM{}

%---- BOX 2 ------------------------------------------------------------
%
% Type below a concise abstract of your proposal (up to 9 lines).
% This parameter is NOT checked at the pdfLaTeX compilation.
%
% DO NOT USE ANY TEX/LATEX MACROS IN THE ABSTRACT

\Abstract{This is a concise abstract of the proposal which may have up
  to 9 lines.}

%---- BOX 3 ------------------------------------------------------------
%
% Description of the observing run(s) necessary for the completion of
% your programme.  The macro takes ten parameters: run ID, period,
% instrument, time requested, month preference, moon requirement,
% seeing requirement, transparency requirement, observing mode and 
% run type.
%
% 1. RUN ID
% Valid values: A, B, ..., Z
% Please note that only one run per intrument is allowed for APEX
%
% 2. PERIOD
% Valid values: 99
% Exceptions:
% Monitoring Programmes: These programmes can span up to four periods.
%
% VLT-XMM proposals: These are only accepted in odd periods and are 
% also valid for the next period.
%
% This parameter is NOT checked at the pdfLaTeX compilation.
%
% 3. INSTRUMENT
% Valid values: AMBER ARTEMIS EFOSC2 FLAMES FLASH FORS2 GRAVITY HARPS HAWKI KMOS LABOCA MUSE NACO OMEGACAM PIONIER SEPIA SHFI SINFONI SOFI SOFOSC SPHERE Special3.6 SpecialAPEX SpecialNTT UVES VIMOS VIRCAM VISIR XSHOOTER
% 
% Please note that only a subset of these instruments will be accepted
% for Monitoring Programmes. Please see the Call for Proposals and the
% ESOFORM User Manual for more details.
%
% 4. TIME REQUESTED
% In hours for Service Mode, in nights for Visitor Mode.
% In either case the time can be rounded up to  1 decimal place. 
% This parameter is NOT checked at the pdfLaTeX compilation.
% 
% 5. MONTH PREFERENCE
% Valid values: apr, may, jun, jul, aug, sep, any
%
% 6. MOON REQUIREMENT
% Valid values: d, g, n
%
% 7. SEEING REQUIREMENT
% Valid values: 0.4, 0.6, 0.8, 1.0, 1.2, 1.4, n
%
% 8. TRANSPARENCY REQUIREMENT
% Valid values: CLR, PHO, THN
%
% 9. OBSERVING MODE
% Valid values: v, s
%
% 10. RUN TYPE
% Valid values: TOO 
% For all Normal & Calibration Programmes this field should be blank.
% For TOO & GTO Programmes, users can specify TOO runs.
% If the field is left blank a default normal, non-TOO run is assumed.
% If a TOO run is specified please make sure that you fill in the TOO page.



\ObservingRun{A}{101}{FORS2}{4h}{may}{n}{0.8}{PHO}{s}{}
%\ObservingRun{A/alt}{99}{FORS2}{3n=2x1+2H2}{may}{n}{0.8}{PHO}{v}{}
%\ObservingRun{B}{99}{VIMOS}{2n=2x1}{jun}{n}{0.6}{CLR}{v}{}
%\ObservingRun{C}{99}{EFOSC2}{3n}{aug}{n}{0.8}{THN}{v}{}
%\ObservingRun{D}{99}{NACO}{0.4n}{may}{n}{0.8}{THN}{v}{}
%\ObservingRun{E}{99}{VIMOS}{1h}{apr}{n}{1.4}{THN}{s}{}
%\ObservingRun{F}{99}{VIMOS}{1h}{apr}{n}{n}{THN}{s}{}

% Proprietary time requested.
% Valid values: % 0, 1, 2, 6, 12

\ProprietaryTime{12}

%---- BOX 4 ------------------------------------------------------------
%
% Indicate below the telescope(s) and number of nights/hours already
% awarded to this programme, if any.
% This macro is optional and can be commented out.
% It is also NOT checked at the pdfLaTeX compilation.

\AwardedNights{NTT}{4n in 97.B-1234}

% Indicate below the telescope(s) and number of nights/hours still
% necessary, in the future, to complete this programme, if any.
% This macro is optional and can be commented out.
% It is also NOT checked at the pdfLaTeX compilation.

\FutureNights{UT2}{20h}

%---- BOX 5 ------------------------------------------------------------
%
% Take advantage of this box to provide any special remark  (up to three
% lines). In case of coordinated observations with XMM, please specify
% both the ESO period and the preferred month for the XMM
% observations here.
% This macro is optional and can be commented out.
% It is also NOT checked at the pdfLaTeX compilation.

\SpecialRemarks{This macro is optional and can be commented out. }
  
%---- BOX 6 ------------------------------------------------------------
% Please provide the ESO User Portal username for the Principal
% Investigator (PI) in the \PI field.
%
% For the Co-I's (CoI) please fill in the following details:
% First and middle initials, family name, the institute code
% corresponding to their affiliation. 
% The corresponding affiliation should be entered for EACH
% Co-I separately in order to ensure the correct details of 
% all Co-I's are stored in the ESO database.
% You can find all institute codes listed according to country
% on the following webpage:
% http://www.eso.org/sci/observing/phase1/countryselect.html
%
% For example, if the Co-I's full name is David Alan William Jones,
% his affiliation is the Observatoire de Paris, Site de Paris, 
% you should write:
% \CoI{D.A.W.}{Jones}{1588}
% Further examples are shown below.
% DO NOT USE ANY TEX/LATEX MACROS HERE
%

\PI{JSMITH999} 
% Replace with PI's ESO User Portal username.

\CoI{L.}{Ma\c con}{1098}
\CoI{R.}{Men\'endez}{1098}
\CoI{S.}{Bailer-Brown}{1154}
\CoI{K.L.}{Giorgi}{1339}
\CoI{S.}{Lichtman}{1377}

% Please note: 
% Due to the way in which the proposal receiver system parses
% the CoI macro, the number of pairs of curly brackets '{}'
% in this macro MUST be strictly equal to 3, i.e., the
% number of parameters of the macro. Accordingly, curly
% brackets should not be used within the parameters (e.g.,
% to protect LaTeX signs).
%
% For instance:
% \CoI{L.}{Ma\c con}{1098}
% \CoI{R.}{Men\'endez}{1098}
%
% are valid, while
%
% \CoI{L.}{Ma{\c}con}{1098}
% \CoI{R.}{Men{\'}endez}{1098}
%
% are not. Unfortunately the receiver does not give an
% explicit error message when such invalid forms are
% used in the CoI macro, but the processing of the proposal
% keeps hanging indefinitely.


%%%%%%%%%%%%%%%%%%%%%%%%%%%%%%%%%%%%%%%%%%%%%%%%%%%%%%%%%%%%%%%%%%%%%%%%
%%%%% THE TWO PAGES OF THE SCIENTIFIC DESCRIPTION AND FIGURES %%%%%%%%%%
%%%%%%%%%%%%%%%%%&&&%%%%%%%%%%%%%%%%%%%%%%%%%%%%%%%%%%%%%%%%%%%%%%%%%%%%
%
%---- BOX 7 ------------------------------------------------------------
%
%               THIS DESCRIPTION IS RESTRICTED TO TWO PAGES 
%
%   THE RELATIVE LENGTHS OF EACH OF THE SECTIONS ARE VARIABLE,
%   BUT THEIR SUM (INCLUDING FIGURES & REFS.) IS RESTRICTED TO TWO PAGES
%
% All macros in this box are NOT checked at the pdfLaTeX compilation.

\ScientificRationale{
  
{\bf The case for AGN feedback:} Determining how today's galaxies have grown and evolved to their present state is the primary goal of extragalactic research. It is now clear that galaxy growth is strongly regulated by so-called ``feedback'' processes (e.g., Vogelsberger et al. 2014, Schaye etal. 2015). Among the most important of these is the suppression of galaxy growth by Active Galactic Nuclei (AGNs) that heat and/or expel gas which would otherwise collapse to form stars (see Fabian, 2012; Harrison, 2017 for reviews).  
  \vspace{1.0mm}
  
  The case for AGN feedback has recently received significant  empirical support from observations. Firstly, X-ray observations of  nearby clusters have revealed AGNs injecting considerable amounts of  energy into the intergalactic medium, preventing it from cooling and  forming stars (e.g., McNamara \& Nulsen 2012). Secondly, there is  clear evidence of fast (i.e., ${\rm >500s~km~s^{-1}}$), ionised  outflows in the optical and near-infrared spectra of a significant  fraction (i.e., $\sim20\%$; e.g., Mullaney et al. 2013; Harrison et  al. 2014) of all AGNs. Both provide strong evidence of energy  transport from AGNs, as required by feedback models.  The problem we  face, however, is that we do still not understand (a) how AGNs are  triggered and (b) what mechanism drives the outflows that transport  energy from the AGN. Until these questions are addressed, it will  remain impossible to test whether AGN feedback is being accurately  implemented in models of galaxy growth.

  \vspace{1.0mm}

   {\bf The role of IFU surveys in studies of AGN feedback:} The key to determining how AGNs are triggered are spatially   resolved kinematics of their host galaxies. This is because any   triggering mechanism must funnel gas to fuel the AGN from galactic   scales to the nucleus. Similarly, to determine how AGN outflows are   driven requires spatially resolved observations of those outflows,   mapped-out by their gas kinematics. Such spatially-resolved   kinematics of the outflows and, in particular, the host galaxy can   {\it only} be delivered by integral field (IFU) observations of   {\it low redshift} (i.e., $z<0.4$) AGNs.

   \vspace{1.0mm}

  To date, detailed IFU observations of nearby AGNs have almost exclusively focussed on radio-weak AGNs (e.g., the CARS survey $[$PI: Husemann$]$). However, {\bf there is strong evidence of an association between fast, ionised outflows and radio-powerful AGNs}, with 50\% of AGNs with 1.4~GHz  radio luminosities ($L_{\rm 1.4GHz}$) above ${\rm 10^{23}~W~Hz^{-1}}$ displaying evidence of powerful (${\rm >500~km~s^{-1}}$) outflows, compared to just 10\% of those  below this radio luminosity threshold (Mullaney et al. 2013; Fig. 1,  {\it left}). Furthermore, our own resolved long-slit and IFU  observations of nearby radio AGNs show clear signs of interaction between radio jets and the ISM (e.g., Holt et al. 2008, Tadhunter et  al. 2014, Santoro et al. 2015). This raises the prospect that jets launched from radio AGNs are an important driver of powerful gas outflows, as predicted by the hydrodynamical simulations of jet/gas interactions described in Wagner et al. (2011; Fig. 1 {\it right};  also Mukherjee et al. 2016). Furthermore, radio loud AGNs are the {\it only} type capable of inducing ``radio-mode'' AGN feedback, which is thought to heat intergalactic gas, preventing it from  collapsing onto galaxies to form stars (e.g., Bower et  al. 2006). Since radio powerful AGNs have largely been avoided by IFU studies, it is likely that they have ignored one of the most important drivers of AGN feedback. We now propose to address this with deep MUSE observations of two prototypical nearby radio AGN. As well as delivering excellent science in its own right, the proposed observations will form a feasibility study for a Large Programme, as recommended by the OPC in P101A.}

  \ImmediateObjective{We propose to obtain deep MUSE observations of two prototypical, nearby radio-powerful AGN to determine: {\bf (a) how they are triggered} and {\bf (b) what drives their outflows}. For $(a)$ our own detailed morphological studies of the 2~Jy sample have revealed that tidal tails, fans, shells, and bridges -- features commonly associated with galaxy mergers -- occur far more frequently around powerful radio AGNs compared to luminosity-matched comparison early-type galaxies (94\% of radio AGNs vs 50\% of non-active early-types when of similar surface brightness; Ramos Almeida et al. 2011, 2012; Fig. 2). However, with both major and minor mergers
  capable of producing strong tidal features, it is not clear what
  type of mergers trigger radio AGN. Thankfully, stellar kinematics can
  be used to distinguish between the types of mergers that have taken
  place. Shallow, long-slit observations give provide tantalising, yet unconfirmed, evidence that radio AGNs are associated with fast rotators. If confirmed, his would connect powerful radio AGNs to major wet mergers, such as those that trigger ULIRGs. By contrast, should they instead be triggered by accretion of hot halo gas or minor mergers as suggested by some models (e.g., King \& Pringle 2006; Hopkins \& Quataert 2010), they would show a preference toward slowly-rotating early-type galaxies. {\bf By mapping the stellar kinematics across the entire host galaxy, our MUSE observations
  will determine what type of galaxy interaction are required to
  trigger powerful radio AGNs}.
      
  %Our two target AGNs are selected from the 2~Jy sample of southern (Dec$<+10$\deg), unbeamed (i.e., non-Blazar) radio powerful AGNs. Over the past two decades, our team has led numerous campaigns to obtain extensive coverage of the southern 2~Jy sample across the full observable electromagnetic spectrum. The 2Jy is unique in terms of the completeness of its multi-wavelength data that includes X-ray imaging/spectroscopy ({\it Chandra}, XMM), optical spectroscopy (ESO 3.6m \& VLT), optical imaging ({\it Gemini}), near-IR imaging (ESO NTT \& VLT), mid to far-IR photometry ({\it Spitzer} \& {\it Herschel}), mid-IR spectroscopy ({\it Spitzer}) and radio imaging (VLA and ATCA); {\it it is now the best observed of any sample of radio-loud AGN}.  {\bf The 2~Jy sample therefore represents our best possible opportunity to determine what processes trigger powerful radio AGNs and drive their outflows.}  

\vspace{1.0mm}

{\bf \large The Observations}

In this section we highlight how our MUSE observations of the 2~Jy
sample will build upon our current understanding of radio AGNs,
enabling us to determine how radio AGNs are triggered, how their
outflows are driven, and measure the properties and impact of these
outflows.

% Radio AGN fall into two broad categories according to their optical
% emission lines: {\bf Strong Line Radio Galaxies (SLRGs)} and {\bf
%   Weak Line Radio Galaxies (WLRGs)}. SLRGs are the radio-loud
% versions of luminous quasars, and those within the 2~Jy sample
% represent some of the most powerful AGNs in the local Universe (see
% review by Tadhunter et al. 2016). Studies of SLRGs therefore provide
% insights into the triggering of powerful AGNs {\it in general}. By
% contrast, WLRGs show little evidence of a ``traditional'' active
% nucleus at optical wavelengths. Yet, WLRGs transfer vast amounts of
% mechanical energy into their surrounding IGM. The 2~Jy sample
% contains both SLRGs ({\bf \color{red} numbers}) and WLRGs ({\bf
%   \color{red} numbers}) important to determine the root causes of
% each mode of AGN feedback.

\vspace{1.0mm}

{\bf How are radio AGNs triggered?:} To establish what triggers
radio AGNs demands detailed observations of their immediate
surroundings, and in particular their host galaxies. Initial
inspection reveals that the majority of radio AGNs reside in massive
($>10^{11}~{\rm M_\odot}$), early type galaxies (e.g., Matthews,
Morgan \& Schmidt 1964). However, our own detailed morphological
studies of the 2~Jy sample have revealed that features such as tidal
tails, fans, shells, and bridges occur far more frequently around powerful
radio AGNs compared to luminosity-matched comparison early-type
galaxies (94\% of radio AGNs vs 50\% of non-active early-types when
of similar surface brightness; Ramos Almeida et al. 2011, 2012;
Fig. 2). This indicates a far more violent {\it recent} merger
history than their early-type morphologies initially suggest.

\vspace{1.0mm}

The prevalence of tidal features around radio AGNs provides clear
evidence that galaxy interactions play a role in triggering at least
some powerful radio AGNs. However, with both major and minor mergers
capable of producing strong tidal features, it is not clear what
type of merger trigger radio AGN. Thankfully, stellar kinematics can
be used to distinguish between the types of mergers that have taken
place. Fast rotating early-type galaxies are produced by wet, major
mergers, while slow rotators are thought to be produced by dry major
or a series of minor mergers (Cappellari et al. 2016 and references
therein).

\vspace{1.0mm}

To date, only a handful of radio AGNs have had their hosts' stellar
kinematics measured, and even then only with long-slit spectra out
to only a fraction of an effective radius (Smith and Heckman et
al. 1990, Bettoni et al. 2001). These few shallow observations
provide tantalising, yet unconfirmed, evidence that radio AGNs are
associated with fast rotators. If confirmed, this would connect
powerful radio AGNs to major wet mergers, possibly representing a
post-starburst phase, since local Ultra Luminous Infrared Galaxies
(ULIRGs) are also associated with fast rotation (e.g., Genzel et
al. 2001; Tacconi et al. 2002). Should they instead be triggered by
an alternative mechanism, such as accretion of hot halo gas or minor
mergers as suggested by some models (e.g., King \& Pringle 2006;
Hopkins \& Quataert 2010), they would show a preference toward
slowly-rotating early-type galaxies. Finally, should radio AGNs show
no preference to either fast or slow rotators, it would imply that
the type of merger is not important, with either wet, dry, major or
minor capable of triggering them. {\bf By mapping the stellar
kinematics across the entire host galaxy, our MUSE observations
will determine what type of galaxy interaction are required to
trigger powerful radio AGNs.}

\vspace{1.0mm}

With our MUSE observations, we will spatially resolve (to at least
one effective radius) the host galaxies of all our sample. We will
measure the off-nuclear stellar velocity shift ($V$) and velocity
dispersion ($\sigma$) from stellar absorption lines. At the low
redshifts of these sources, the MUSE spectra will cover the strong
Mg{\it b}$\lambda$5200 absorption line, although we will fit the
whole stellar continuum with gaussian-convolved stellar templates to
maximise the information from the stellar continuum (i.e., excluding
emission lines and using the Penalised Pixel-Fitting method of
Cappellari \& Emsellem, 2004). To extract just the first two
velocity modes (i.e., $V$ and $\sigma$) requires a comparitavely low
continuum signal-to-noise of $\sim$10 (e.g., Cappellari et
al. 2007). To ensure we measure the characteristic kinematics of the
whole galaxy, rather than just the core, {\it it is vital that we
reach this sensitivity to at least one effective radius} (a
diameter of 20 kpc, or 20\arcsec\ [4\arcsec ] for our lowest
[highest] redshift target). This can only be achieved with deep
($\sim2~{\rm hr}$) observations (see Box 9).

\vspace{1.0mm}

We will use the criteria of Emsellem et al. (2007) to distinguish
between fast and slow rotating early type radio AGN hosts. This uses
the resolved $V$ and $\sigma$ maps to overcome some of the
degeneracies associated with traditional $V/\sigma$
measurements. Next, we will compare the relative numbers of fast and
slow rotators in our sample against mass-matched samples of non-AGN
early type galaxies from the {\sc Atlas}$^{\rm 3D}$ survey
(Cappellari et al. 2011). Should our sample display a significant
departure from the relative numbers of fast to slow rotators in this
comparison sample, it would imply a preference of one over the
other. We will also compare against the radio-weak AGNs targeted with
MUSE as part of the Close AGN Reference Survey (CARS; PI: Husemann)
to determine whether powerful radio AGNs show evidence of being
triggered via different mechanisms compared to radio-weak AGNs. With
36 radio AGNs, our sample contains sufficient statistics to robustly
test for this as a function of AGN type (broad line, narrow line,
weak line) and luminosity, thereby demonstrating whether different
classes of AGNs are triggered by different mechanisms.

\vspace{1.0mm}

{\bf What drives outflows from radio AGNs?:} To have any impact on
galaxy evolution, the energy from an AGN, radio powerful or
otherwise, must be transmitted into their host galaxies or
surrounding material {\it and} effect some influence. Our team has
played a leading role in identifying fast, yet non-relativistic
(${\rm \sim1000~km\ s^{-1}}$) winds associated with powerful AGNs
(e.g., Tadhunter et al. 2001, 2014; Alexander et al. 2008;
Villar-Martin et al. 2014, 2016; Harrison et al. 2014; Santoro et
al. 2015; Spence et al. 2016; Figs. 3 \& 4). These winds are often
extended over kpc-scales, thereby {\it potentially} affecting a
large fraction of the host galaxy as demanded by AGN feedback models
(e.g., Alexander et al, 2008; Harrison et al. 2012, 2014; Tadhunter
et al. 2014). Our investigations have also found that such winds are
more prevalent among AGNs displaying evidence of nuclear radio
emission.  Indeed, our long-slit observations of a subsample of the
2~Jy sample has shown that some of the fastest AGN winds in the
local Universe are associated with powerful radio AGNs (e.g.,
Tadhunter et al. 2016). As such, it is important for our models of
feedback that we establish their primary driving mechanism, whether
the radio jet or the intense radiation from the AGN. {\bf Our MUSE
observations will achieve this by mapping the winds in the 2~Jy
AGN, allowing us to relate them to the resolved radio jets.}

\vspace{1.0mm}

By measuring the profiles of the strong forbidden $[$O~{\sc
iii}$]\lambda5007$ emission line, we will map the kinematics of
the ionised gas in our sample. The $[$O~{\sc iii}$]$ line traces low
density gas and has been used extensively in recent studies, not
least our own, to successfully measure the kinematics and extent of
outflows in AGN host galaxies. The ionised phase, however, only
represents a fraction of the gas in a galaxy, so we will use the
Na~{\sc i} $\lambda\lambda5890,5896$ absorption lines to map the
kinematics of the neutral gas phase (e.g., Rupke \& Veilleux,
2013). {\it It is important that we have sufficiently deep
observations to measure this absorption line against the stellar
continuum}.  Any Na~{\sc i} absorption intrinsic to the stellar
continuum will be modelled using stellar templates. At every
position in the galaxy, we will decompose the $[${\rm O}~{\sc
iii}$]$ and Na~{\sc i} lines into their various kinematic
components, identifying those regions displaying the strongly
shifted or broad kinematic components characteristic of powerful AGN
outflows. By mapping the stellar and gas kinematics across the whole
galaxy, {\it our deep observations will be able to identify even
small departures from gravitational motions, and thus be highly
sensitive to extended outflows}. We will then relate the location
of the outflows to the positions of the radio jets that have already
been mapped-out at high (i.e., sub-arcsecond) resolutions with our VLA
and ATCA observations. {\it Should any AGN show outflows with a
significantly wider opening angle than the jets, it would imply
radiative driving in that case.}

\vspace{1.0mm}

{\bf What effect do AGN outflows have on their host galaxies?:} As
well as establishing the mechanism driving AGN winds, to test AGN
feedback models we must also determine whether they have any real
effect on their host galaxies. The first step in doing this is to
measure the energy content of the wind and (i) compare it to model
predictions and (ii) empirically assess whether it is sufficient to
evacuate the galaxy of significant fraction of its gas content. The
second step is to determine whether the winds are having a direct
influence on the host by comparing the rates of star formation
within the winds against those in the rest of the galaxy. All of the
2~Jy sample have {\it Spitzer} and {\it Herschel} coverage,
providing obscuration-independent star formation rates
(SFRs). However, the host is only resolved at these infrared
wavelengths in a small minority of cases. For the others, we will
use the MUSE spectra to produce resolved BPT diagnostics to map-out
regions of star-formation within the host galaxies and, following
e.g., Perna et al. (2015) \& Maiolino et at. (2017), relate these
regions to the resolved AGN outflows. {\bf By mapping the source of
ionising radiation (AGN, star formation), and density and
kinematics of the outflows, our MUSE observations will provide the
information needed to comprehensively measure the impact of radio
AGNs on their host galaxies.}

\vspace{1.0mm}  

To measure the energy content of the extended outflows requires
knowledge of their mass content. Thirty-five of our sample are at
low enough redshifts that the $[$S~{\sc
ii}$]\lambda\lambda6717,6732$ doublet falls within the spectral
range of MUSE, enabling us to use this density-sensitive line for
mass measurements. This density information, combined with the
kinematics and extents of the outflows, will enable us to determine
the mass and mass flux of the outflows. From this, we will measure
the energy and momentum content of the outflows. This will be
compared against model predictions (e.g., Zubovas \& King 2012;
Gabor \& Bournaud 2014) and the gravitational potential of the
galaxy (measured from stellar and non-outflowing gas kinematics) to
determine whether the outflows contain sufficient energy to have a
significant impact on host.

\vspace{1.0mm}  

Finally, we will use the approach of using resolved BPT diagnostics
outlined in e.g., Perna et al. (2015); Maiolino et al. (2017) to
directly measure the impact of the outflows on star formation within
the extended outflows. This method solves the problem that the
H$\beta$, $[$O~{\sc iii}$]\lambda5007$, $[$O~{\sc
iii}$]\lambda\lambda6548, 6584$ and H$\alpha$ lines (i.e., the
traditional emission lines used in BPT diagnostics) are dominated by
AGN ionisation by kinematically decomposing the emission lines into
separate AGN-ionised and star formation-ionised components. By
relating regions of star-formation to the locations of the outflows,
we will measure the impact of these outflows on star formation
within the host galaxy (whether they enhance [positive feedback] or
suppress [negative feedback] star-formation). Since BPT diagnostics
are sensitive to star formation within the past $\sim$10~Myr, this
provides a measurement of the immediate effect of AGN feedback
(whether positive or negative) on short timescales before the AGN
has had a chance to ``switch off'' (e.g., Hickox et al. 2014)} 

%
%---- THE SECOND PAGE OF THE SCIENCE CASE CAN INCLUDE FIGURES ----------
%
% Up to ONE page of figures can be added to your proposal.  
% The text and figures of the scientific description must not
% exceed TWO pages in total. 
% If you use color figures, do make sure that they are still readable
% if printed in black and white. Figures must be in PDF or JPEG format.
% Each figure has a size limit of 1MB.
% MakePicture and MakeCaption are optional macros and can be commented out.

\MakePicture{galaxy.pdf}{angle=90,width=10cm}
\MakeCaption{Fig.~1: A caption for your figure can be inserted here.}

\MakeCaption{References can also be included using MakeCaption.
  For example:

  References: }

%%%%%%%%%%%%%%%%%%%%%%%%%%%%%%%%%%%%%%%%%%%%%%%%%%%%%%%%%%%%%%%%%%%%%%
%%%%% THE PAGE OF TECHNICAL JUSTIFICATIONS %%%%%%%%%%%%%%%%%%%%%%%%%%%%%
%%%%%%%%%%%%%%%%%%%%%%%%%%%%%%%%%%%%%%%%%%%%%%%%%%%%%%%%%%%%%%%%%%%%%%%%
%
%---- BOX 8 ------------------------------------------------------------
%
% Provide below a careful justification of the requested lunar phase
% and of the requested number of nights or hours.  
% All macros in this box are NOT checked at the pdfLaTeX compilation.

\WhyLunarPhase{Provide here a careful justification of the requested
  lunar phase.}  

\WhyNights{Provide a careful justification of the requested number of 
  nights or hours for each observing run here. ESO Exposure Time 
  Calculators exist for all Paranal and La Silla instruments and are 
  available at the following web address: 

  http://www.eso.org/observing/etc .

  Links to exposure time calculators for APEX instrumentation
  can be found in Section 7 of the Call for Proposals.
}

\TelescopeJustification{Justification for the use of the selected
  telescope (e.g., VLT, APEX, etc...)  with respect to other 
  available alternatives.}

\ModeJustification{Explain if a particular observing mode is specifically needed for this programme. If either can, in principle, be used then please enter N/A. }


% Please specify the type of calibrations needed.
% In case of special calibration the second parameter is used to enter 
% specific details.
% Valid values: standard, special
\Calibrations{special}{Adopt a special calibration}
%\Calibrations{standard}{}


%%%%%%%%%%%%%%%%%%%%%%%%%%%%%%%%%%%%%%%%%%%%%%%%%%%%%%%%%%%%%%%%%%%%%%%
%% PAGE OF BOXES 9-10  %%%%%%%%%%%%%%%%%%%%%%%%%%%%%%%%%%%%%%%%%%%%%%%%
%%%%%%%%%%%%%%%%%%%%%%%%%%%%%%%%%%%%%%%%%%%%%%%%%%%%%%%%%%%%%%%%%%%%%%%
%
%---- BOX 9 -- Use of ESO Facilities --------------------------------
%
% This macro is optional and can be commented out.
% It is also NOT checked at the pdfLaTeX compilation.
% LastObservationRemark: Report on the use of the ESO facilities during
%  the last 2 years (4 observing periods). Describe the status of the
%  data obtained and the scientific output generated.

\LastObservationRemark{This macro is optional and can be commented out.}

%
%---- BOX 9a -- ESO Archive ------------------------------------------
%
% Are the data requested in this proposal in the ESO Archive
% (http://archive.eso.org)? If yes, explain the need for new data.
% This macro is NOT checked at the pdfLaTeX compilation.

\RequestedDataRemark{Are the data requested in this proposal in the
  ESO Archive (http://archive.eso.org)? If yes, explain the need for
  new data.}

%
%---- BOX 9b -- ESO GTO/Public Survey Programme Duplications---------
%
% If any of the targets/regions in ongoing GTO Programmes and/or
% Public Surveys are being duplicated here, please explain why.
% This macro is optional and can be commented out.
% It is also NOT checked at the pdfLaTeX compilation.

\RequestedDuplicateRemark{
  Specify whether there is any duplication of targets/regions covered 
  by ongoing GTO and/or Public Survey programmes. If so, please 
  explain the need for the new data here. Details on the protected 
  target/fields in these ongoing programmes can be found at: 

  GTO programmes: http://www.eso.org/sci/observing/teles-alloc/gto.html
  
  Public Survey programmes: 
  http://www.eso.org/sci/observing/PublicSurveys/sciencePublicSurveys.html
  
  This macro is optional and can be commented out.
}

%
%---- BOX 10 ------ Applicant(s) publications ---------------------
%
% Applicant's publications related to the subject of this proposal
% during the past two years.  Use the simplified abbreviations for
% references as in A&A.  Separate each reference with the following
% usual LaTex command: \smallskip\\
%   
%   Name1 A., Name2 B., 2001, ApJ, 518, 567: Title of article1
%   \smallskip\\
%   Name3 A., Name4 B., 2002, A\&A, 388, 17: Title of article2
%   \smallskip\\
%   Name5 A. et al., 2002, AJ, 118, 1567: Title of article3
%
% This macro is NOT checked at the pdfLaTeX compilation.

\Publications{
  Name1 A., Name2 B., 2001, ApJ, 518, 567: Title of article1
  \smallskip\\
  Name3 A., Name4 B., 2002, A\&A, 388, 17: Title of article2
  \smallskip\\
  Name5 A. et al., 2002, AJ, 118, 1567: Title of article3
}

%%%%%%%%%%%%%%%%%%%%%%%%%%%%%%%%%%%%%%%%%%%%%%%%%%%%%%%%%%%%%%%%%%%%%%%%
%%%%% THE PAGE OF THE TARGET/FIELD LIST %%%%%%%%%%%%%%%%%%%%%%%%%%%%%%%%
%%%%%%%%%%%%%%%%%%%%%%%%%%%%%%%%%%%%%%%%%%%%%%%%%%%%%%%%%%%%%%%%%%%%%%%%
%
%---- BOX 11 -----------------------------------------------------------
%
% Complete list of targets/fields requested.  The macro takes nine
% parameters: run ID, target field/name, RA, Dec, time on target, magnitude, 
% diameter, additional information, reference star.
%
% 1. RUN ID
% Valid values: run IDs specified in BOX 3
%
% 2. TARGET FIELD/NAME
%
% 3. RA (J2000)
% Format: hh mm ss.f, or hh mm.f, or hh.f
% Use 00 00 00 for unknown coordinates
% This parameter is NOT checked at the pdfLaTeX compilation.
% 
% 4. Dec (J2000)
% Format: dd mm ss, or dd mm.f, or dd.f
% Use 00 00 00 for unknown coordinates
% This parameter is NOT checked at the pdfLaTeX compilation.
%
% 5. TIME ON TARGET
% Format: hours (overheads and calibration included)
% This parameter is NOT checked at the pdfLaTeX compilation.
%
% 6. MAGNITUDE
% This parameter is NOT checked at the pdfLaTeX compilation.
%
% 7. ANGULAR DIAMETER
% This parameter is NOT checked at the pdfLaTeX compilation.
%
% 8. ADDITIONAL INFORMATION
% Any relevant additional information may be inserted here.
% For APEX and CRIRES runs, the requested PWV upper limit MUST
% be specified for each target using this field.
% For APEX runs, the acceptable LST range MUST also be specified here.
% This parameter is NOT checked at the pdfLaTeX compilation.
%
% 9. REFERENCE STAR ID
% See Users' Manual.
% This parameter is NOT checked at the pdfLaTeX compilation.
%
% Long lists of targets will continue on the last page of the
% proposal.
%
%                       ** VERY IMPORTANT ** 
% The scheduling of your programme will take into account ALL targets
% given in this list. INCLUDE ONLY TARGETS REQUESTED FOR P99 
% (except for runs outside P99, e.g. for VLT-XMM and Monitoring proposals).
%
% DO NOT USE ANY TEX/LATEX MACROS FOR THE TARGETS

%\Target{ABC}{Cen A}{13 25 27.61}{-43 01 08.8}{8.0}{7.9}{20 min}{NGC 5128}{}
\Target{A}{NGC 5139}{13 26.8}{-47 29}{5.0}{6.12}{1 deg}{Omega Cen}{}
%\Target{BC}{NGC 6058}{15 12 51.0}{-38 07 33}{15.0}{11.6}{}{plan. neb.}{}
%\Target{B}{M 5}{15 18 33}{+02 04 58}{8.0}{7}{}{glob. cluster}{}
%\Target{C}{M 6}{17 40.1}{-32 13}{10.0}{2.0}{4.3}{Butterfly cl.}{}
%\Target{C}{M 8}{18 03 37}{-24 23.2}{1.0}{3.8}{30 min}{Lagoon neb.}{}
%\Target{C}{NGC 6822}{19 44 57.8}{-14 48 11}{20.0}{18}{}{Barnard's gal.}{}
%\Target{D}{NGC 7793}{23 57 49.9}{-32 35 20}{20.0}{18}{}{Sd gal.}{S322120026}
%\Target{E}{Alpha Ori}{06 45 08.9}{-16 42 58}{1}{-1.4}{6 mas}{Sirius}{}
%\Target{F}{Alpha Ori}{06 45 08.9}{-16 42 58}{1}{-1.4}{6 mas}{Sirius}{}


%                      ***************** 
%                      ** PWV limits **
% For CRIRES and all APEX instruments users must specify the PWV upper
% limits for each target. For example:
%\Target{}{Alpha CMa}{06 45 08.9}{-16 42 58}{1}{-1.4}{6 mas}{PWV=1.0mm, Sirius}{}
%\Target{}{HD 104237}{12 00 05.6}{-78 11 33}{1}{}{}{PWV<0.7mm;LST=9h00-15h00}{}
%
%                      *****************

% Use TargetNotes to include any comments that apply to several or all
% of your targets.
% This macro is NOT checked at the pdfLaTeX compilation.

\TargetNotes{A note about the targets and/or strategy of selecting the targets during the run. For APEX runs please remember to specify the PWV limits for each target under 'Additional info' in the table above.}

%%%%%%%%%%%%%%%%%%%%%%%%%%%%%%%%%%%%%%%%%%%%%%%%%%%%%%%%%%%%%%%%%%%%%%%%
%%%%% TWO PAGES OF SCHEDULING REQUIREMENTS %%%%%%%%%%%%%%%%%%%%%%%%%%%%%
%%%%%%%%%%%%%%%%%%%%%%%%%%%%%%%%%%%%%%%%%%%%%%%%%%%%%%%%%%%%%%%%%%%%%%%%
%
%---- BOX 12 -----------------------------------------------------------
%

% Uncomment the following line if the proposal involves time-critical
% observations, or observations to be performed at specific time
% intervals. Please leave these brackets blank. Details of time
% constraints can be entered in Special Remarks and using the
% other flags in Box 13.
%
%
\HasTimingConstraints{}

%
% The timing constraint macros listed below 
% are optional and can be commented out:
% \HasTimingConstraints, \RunSplitting, \Link and \TimeCritical
% They are also NOT checked at the pdfLaTeX compilation.


% 1. RUN SPLITTING, FOR A GIVEN ESO TELESCOPE (Visitor Mode only)
%
% 1st argument: run ID
% Valid values: run IDs specified in BOX 3
%
% 2nd argument: run splitting requested for sub-runs
% This parameter is NOT checked at the pdfLaTeX compilation.

%\RunSplitting{B}{1,10s,1}
%\RunSplitting{C}{2,10s,2,20w,2,15s,4H2}


% 2. LINK FOR COORDINATED OBSERVATIONS BETWEEN DIFFERENT RUNS.
%\Link{B}{after}{A}{10}
%\Link{C}{after}{B}{}
%\Link{E}{simultaneous}{F}{}

% 3. UNSUITABLE PERIOD(S) OF TIME
%
% 1st argument: run ID
% Valid values: run IDs specified in BOX 3
%
% 2nd argument: Chilean start date for the unsuitable time
% Format: dd-mmm-yyyy
% This parameter is NOT checked at the pdfLaTeX compilation.
%
% 3rd argument: Chilean end date for the unsuitable time
% Format: dd-mmm-yyyy
% This parameter is NOT checked at the pdfLaTeX compilation.

\UnsuitableTimes{A}{15-jul-17}{18-jul-17}{Insert explanation of unsuitable time here.}
%\UnsuitableTimes{B}{15-jul-17}{18-jul-17}{Insert explanation of unsuitable time here.}
%\UnsuitableTimes{C}{20-jul-17}{23-jul-17}{Insert explanation of unsuitable time here.}


%
%---- BOX 12 contd.. -- Scheduling Requirements 
%

% SPECIFIC DATE(S) FOR TIME-CRITICAL OBSERVATIONS
% Please note: The dates must correspond to the LOCAL CHILEAN observing dates.
%
% The 2nd and 3rd parameters are NOT checked at the pdfLaTeX compilation.
% 1st argument: run ID
% Valid values: run IDs specified in BOX 3
%
% 2nd argument: Chilean start date for the critical period.
% Format: dd-mmmm-yyyy 
%
% 3rd argument: Chilean end date for the critical period.
% Format: dd-mmmm-yyyy

\TimeCritical{A}{12-may-17}{14-may-17}{Insert reason for time-critical observations.}
%\TimeCritical{D}{1-may-17}{2-may-17}{Insert reason for time-critical observations.}
%\TimeCritical{D}{17-may-17}{18-may-17}{Insert reason for time-critical observations.}
%\TimeCritical{D}{23-may-17}{24-may-17}{Insert reason for time-critical observations.}



%%%%%%%%%%%%%%%%%%%%%%%%%%%%%%%%%%%%%%%%%%%%%%%%%%%%%%%%%%%%%%%%%%%%%%%%
%
%---- BOX 14 -----------------------------------------------------------
%
% INSTRUMENT CONFIGURATIONS:
%
% Uncomment only the lines related to instrument configuration(s)
% needed for the acquisition of your planned observations.
%
% 1st argument: run ID
% Valid values: run IDs specified in BOX 3
%
% 2nd argument: instrument
% This parameter is NOT checked at the pdfLaTeX compilation.
%
% 3rd argument: mode
% This parameter is NOT checked at the pdfLaTeX compilation.
% Please note that RRM mode is only available for some specific
% instrument configurations. 
%
% 4th argument: additional information
% This parameter is NOT checked at the pdfLaTeX compilation.
%
% All parameters are mandatory and cannot be empty. Do NOT specify
% Instrument Configurations for alternative runs.

% Examples (to be commented or deleted)

\INSconfig{A}{FORS2}{Detector}{MIT}
\INSconfig{A}{FORS2}{IMG}{ESO filters: provide list HERE}
%\INSconfig{B}{VIMOS}{IFU 0.33"/fibre}{LR-Blue}
%\INSconfig{C}{EFOSC2}{Imaging-filters}{EFOSC2 filters: provide list here}
%\INSconfig{D}{NACO}{IMG 54 mas/px VIS-WFS}{provide list of filters HERE}
%\INSconfig{E}{VIMOS}{IFU 0.33"/fibre}{LR-Blue}
%\INSconfig{F}{VIMOS}{IFU 0.33"/fibre}{LR-Blue}
%
% Real list of instrument configurations

%%%%%%%%%%%%%%%%%%%%%%%%%%%%%%%%%%%%%%%%%%%%%%%%%%%%%%%%%%%%%%%%%%%%%%%%%
% Paranal
%
%-----------------------------------------------------------------------
%---- NAOS/CONICA at the VLT-UT1 (ANTU)  -------------------------------
%-----------------------------------------------------------------------
%
%\INSconfig{}{NACO}{PRE-IMG}{provide list of filters HERE}
%
% Specify the NGS name, distance from target and magnitude  
%(Vmag preferred, otherwise Rmag) in the target list,
% and uncomment the following line
%\INSconfig{}{NACO}{NGS}{-}
%
%\INSconfig{}{NACO}{Special Cal}{Select if you have special calibrations}
%\INSconfig{}{NACO}{Pupil Track}{Select if you need pupil tracking mode}
%\INSconfig{}{NACO}{Cube}{Select if you need cube mode}
%
%\INSconfig{}{NACO}{SAM VIS-WFS}{Provide list of masks and filters HERE}
%\INSconfig{}{NACO}{SAM IR-WFS}{Provide list of masks and filters HERE}
%\INSconfig{}{NACO}{SAMPol VIS-WFS}{Provide list of masks and filters HERE}
%\INSconfig{}{NACO}{SAMPol IR-WFS}{Provide list of masks and filters HERE}
%
%\INSconfig{}{NACO}{IMG 54 mas/px IR-WFS}{provide list of filters HERE}
%\INSconfig{}{NACO}{IMG 27 mas/px IR-WFS}{provide list of filters HERE}
%\INSconfig{}{NACO}{IMG 13 mas/px IR-WFS}{provide list of filters HERE}
%\INSconfig{}{NACO}{IMG 54 mas/px VIS-WFS}{provide list of filters HERE}
%\INSconfig{}{NACO}{IMG 27 mas/px VIS-WFS}{provide list of filters HERE}
%\INSconfig{}{NACO}{IMG 13 mas/px VIS-WFS}{provide list of filters HERE}
%
%\INSconfig{}{NACO}{CORONA AGPM VIS-WFS}{provide list of filters (L',NB-3.74,NB-4.05) HERE}
%\INSconfig{}{NACO}{CORONA AGPM IR-WFS}{provide list of filters (L',NB-3.74,NB-4.05) HERE}
%
%\INSconfig{}{NACO}{POL 54 mas/px IR-WFS}{provide list of filters HERE}
%\INSconfig{}{NACO}{POL 27 mas/px IR-WFS}{provide list of filters HERE}
%\INSconfig{}{NACO}{POL 13 mas/px IR-WFS}{provide list of filters HERE}
%\INSconfig{}{NACO}{POL 54 mas/px VIS-WFS}{provide list of filters HERE}
%\INSconfig{}{NACO}{POL 27 mas/px VIS-WFS}{provide list of filters HERE}
%\INSconfig{}{NACO}{POL 13 mas/px VIS-WFS}{provide list of filters HERE}
%
%

%-----------------------------------------------------------------------
%---- FORS2 at the VLT-UT1 (ANTU) --------------------------------------
%-----------------------------------------------------------------------
%If you require the E2V (Blue) detector uncomment the following line
%\INSconfig{}{FORS2}{Detector}{E2V}
%
%If you require the MIT (RED) detector uncomment the following line
%\INSconfig{}{FORS2}{Detector}{MIT}
%
% If you require the High-Resolution  collimator uncomment the following line
%\INSconfig{}{FORS2}{collimator}{HR}
%
% Uncomment the line(s) corresponding to the imaging mode(s) you require and
% provide the list of filters needed  for your observations:
%
%\INSconfig{}{FORS2}{PRE-IMG}{ESO filters: provide list HERE}
%\INSconfig{}{FORS2}{IMG}{ESO filters: provide list HERE}
%\INSconfig{}{FORS2}{IMG}{User's own filters (to be described in text)}
%\INSconfig{}{FORS2}{IPOL}{ESO filters: provide list HERE}
%\INSconfig{}{FORS2}{IPOL}{User's own filters (to be described in text)}
%\INSconfig{}{FORS2}{HIT-MS}{Provide list of grisms HERE}
%
%
% Uncomment the line(s) corresponding to the spectroscopic mode(s) you require and
% provide the list of grism+filter combination needed  for your observations:
%
%\INSconfig{}{FORS2}{LSS}{Provide list of grism+filter combinations HERE}
%\INSconfig{}{FORS2}{MOS}{Provide list of grism+filter combinations HERE}
%\INSconfig{}{FORS2}{PMOS}{Provide list of grism+filter combinations HERE}
%\INSconfig{}{FORS2}{MXU}{Provide list of grism+filter combinations HERE}
%\INSconfig{}{FORS2}{HITI}{ESO filters: provide list HERE}
%\INSconfig{}{FORS2}{HIT-OS}{Provide list of grisms HERE}
%
% Uncomment the following line for Rapid Response Mode observations
%
%\INSconfig{}{FORS2}{RRM}{yes}
%
% Uncomment the following line for use of the Virtual Image Slicer
%\INSconfig{}{FORS2}{Virtual Image Slicer}{VM only}
%-----------------------------------------------------------------------
%---- KMOS at the VLT-UT1 (ANTU) ---------------------------------------
%-----------------------------------------------------------------------
%
%\INSconfig{}{KMOS}{IFU}{provide list of settings (IZ, YJ, H, K, HK) here} 
%
%-----------------------------------------------------------------------
%---- FLAMES at the VLT-UT2 (KUEYEN) -----------------------------------
%-----------------------------------------------------------------------
%\INSconfig{}{FLAMES}{UVES}{Specify the UVES setup below}
%\INSconfig{}{FLAMES}{GIRAFFE-MEDUSA}{Specify the GIRAFFE setup below}
%\INSconfig{}{FLAMES}{GIRAFFE-IFU}{Specify the GIRAFFE setup below}
%\INSconfig{}{FLAMES}{GIRAFFE-ARGUS}{Specify the GIRAFFE setup below}
%\INSconfig{}{FLAMES}{Combined: UVES + GIRAFFE-MEDUSA}{Specify the UVES and
%GIRAFFE setups below}
%\INSconfig{}{FLAMES}{Combined: UVES + GIRAFFE-IFU}{Specify the UVES and
%GIRAFFE setups below}
%\INSconfig{}{FLAMES}{Combined: UVES + GIRAFFE-ARGUS}{Specify the UVES and
%GIRAFFe setups below}
%
%
% If you have selected UVES, either standalone or in combined mode,
% please specify the UVES standard setup(s) to be used
%\INSconfig{}{FLAMES}{UVES}{standard setup Red 520}
%\INSconfig{}{FLAMES}{UVES}{standard setup Red 580}
%\INSconfig{}{FLAMES}{UVES}{standard setup Red 580 + simultaneous calibration}
%\INSconfig{}{FLAMES}{UVES}{standard setup Red 860}
%
%\INSconfig{}{FLAMES}{GIRAFFE}{fast readout mode 625kHz VM only}
%\INSconfig{}{FLAMES}{GIRAFFE}{slow readout mode 50kHz VM only}
%
% If you have selected GIRAFFE, either standalone or in combined mode
% please specify the GIRAFFE standard setups(s) to be used
%\INSconfig{}{FLAMES}{GIRAFFE}{standard setup HR01 379.0}
%\INSconfig{}{FLAMES}{GIRAFFE}{standard setup HR02 395.8}
%\INSconfig{}{FLAMES}{GIRAFFE}{standard setup HR03 412.4}
%\INSconfig{}{FLAMES}{GIRAFFE}{standard setup HR04 429.7}
%\INSconfig{}{FLAMES}{GIRAFFE}{standard setup HR05 447.1 A}
%\INSconfig{}{FLAMES}{GIRAFFE}{standard setup HR05 447.1 B}
%\INSconfig{}{FLAMES}{GIRAFFE}{standard setup HR06 465.6}
%\INSconfig{}{FLAMES}{GIRAFFE}{standard setup HR07 484.5 A}
%\INSconfig{}{FLAMES}{GIRAFFE}{standard setup HR07 484.5 B}
%\INSconfig{}{FLAMES}{GIRAFFE}{standard setup HR08 504.8}
%\INSconfig{}{FLAMES}{GIRAFFE}{standard setup HR09 525.8 A}
%\INSconfig{}{FLAMES}{GIRAFFE}{standard setup HR09 525.8 B}
%\INSconfig{}{FLAMES}{GIRAFFE}{standard setup HR10 548.8}
%\INSconfig{}{FLAMES}{GIRAFFE}{standard setup HR11 572.8}
%\INSconfig{}{FLAMES}{GIRAFFE}{standard setup HR12 599.3}
%\INSconfig{}{FLAMES}{GIRAFFE}{standard setup HR13 627.3}
%\INSconfig{}{FLAMES}{GIRAFFE}{standard setup HR14 651.5 A}
%\INSconfig{}{FLAMES}{GIRAFFE}{standard setup HR14 651.5 B}
%\INSconfig{}{FLAMES}{GIRAFFE}{standard setup HR15 665.0}
%\INSconfig{}{FLAMES}{GIRAFFE}{standard setup HR15 679.7}
%\INSconfig{}{FLAMES}{GIRAFFE}{standard setup HR16 710.5}
%\INSconfig{}{FLAMES}{GIRAFFE}{standard setup HR17 737.0 A}
%\INSconfig{}{FLAMES}{GIRAFFE}{standard setup HR17 737.0 B}
%\INSconfig{}{FLAMES}{GIRAFFE}{standard setup HR18 769.1}
%\INSconfig{}{FLAMES}{GIRAFFE}{standard setup HR19 805.3 A}
%\INSconfig{}{FLAMES}{GIRAFFE}{standard setup HR19 805.3 B}
%\INSconfig{}{FLAMES}{GIRAFFE}{standard setup HR20 836.6 A}
%\INSconfig{}{FLAMES}{GIRAFFE}{standard setup HR20 836.6 B}
%\INSconfig{}{FLAMES}{GIRAFFE}{standard setup HR21 875.7}
%\INSconfig{}{FLAMES}{GIRAFFE}{standard setup HR22 920.5 A}
%\INSconfig{}{FLAMES}{GIRAFFE}{standard setup HR22 920.5 B}
%\INSconfig{}{FLAMES}{GIRAFFE}{standard setup LR01 385.7}
%\INSconfig{}{FLAMES}{GIRAFFE}{standard setup LR02 427.2}
%\INSconfig{}{FLAMES}{GIRAFFE}{standard setup LR03 479.7}
%\INSconfig{}{FLAMES}{GIRAFFE}{standard setup LR04 543.1}
%\INSconfig{}{FLAMES}{GIRAFFE}{standard setup LR05 614.2}
%\INSconfig{}{FLAMES}{GIRAFFE}{standard setup LR06 682.2}
%\INSconfig{}{FLAMES}{GIRAFFE}{standard setup LR07 773.4}
%\INSconfig{}{FLAMES}{GIRAFFE}{standard setup LR08 881.7}
%
%\INSconfig{}{FLAMES}{GIRAFFE}{fast readout mode 625kHz VM only}
%
%-----------------------------------------------------------------------
%---- X-SHOOTER at the VLT-UT2 (KUEYEN)
%-----------------------------------------------------------------------
%
%\INSconfig{}{XSHOOTER}{300-2500nm}{SLT}
%\INSconfig{}{XSHOOTER}{300-2500nm}{IFU}
%
% Slits (SLT only):
%
%UVB arm, available slits in arcsec: 0.5, 0.8, 1.0, 1.3, 1.6, 5.0
%VIS arm, available slits in arcsec: 0.4, 0.7, 0.9, 1.2, 1.5, 5.0 
%NIR arm, available slits in arcsec: 0.4, 0.6, 0.6JH, 0.9, 0.9JH, 1.2, 5.0
%  The 0.6JH and 0.9JH include a stray light K-band blocking filter
%  that allow sky limited studies in J and H bands.
%
%The slits for IFU  are fixed and do not need to be mentioned here.
%
% Replace SLIT-UVB, SLIT-VIS, SLIT-NIR with the choice of the slits:
%\INSconfig{}{XSHOOTER}{SLT}{SLIT-UVB,SLIT-VIS,SLIT-NIR}
%
% Detector readout mode:
%
% UVB and VIS arms: available readout modes and binning:
% 100k-1x1, 100k-1x2, 100k-2x2, 400k-1x1, 400k-1x2, 400k-2x2
% The NIR readout mode is fixed  to NDR.
%
%\INSconfig{}{XSHOOTER}{IFU}{readout UVB,readout VIS,readout NIR}
%\INSconfig{}{XSHOOTER}{SLT}{readout UVB,readout VIS,readout NIR}
%
% Imaging mode 
% replace 'list of filters' by the actual filters you wish to use among:
% U, B, V, R, I, Uprime, Gprime, Rprime, Iprime, Zprime
% Please note that the imaging mode can only be used in combination with slit or IFU observations
%\INSconfig{}{XSHOOTER}{IMG}{list of filters}
%
%\INSconfig{}{XSHOOTER}{RRM}{yes}
%
%-----------------------------------------------------------------------
%---- UVES at the VLT-UT2 (KUEYEN) -------------------------------------
%-----------------------------------------------------------------------
%
%\INSconfig{}{UVES}{BLUE}{Standard setting: 346}
%\INSconfig{}{UVES}{BLUE}{Standard setting: 437}
%\INSconfig{}{UVES}{BLUE}{Non-std setting: provide central wavelength  HERE}
%
%\INSconfig{}{UVES}{RED}{Standard setting: 520}
%\INSconfig{}{UVES}{RED}{Standard setting: 580}
%\INSconfig{}{UVES}{RED}{Standard setting: 600}
%\INSconfig{}{UVES}{RED}{Iodine cell standard setting: 600}
%\INSconfig{}{UVES}{RED}{Standard setting: 860}
%\INSconfig{}{UVES}{RED}{Non-std setting: provide central wavelength HERE}
%
%\INSconfig{}{UVES}{DIC-1}{Standard setting: 346+580}
%\INSconfig{}{UVES}{DIC-1}{Standard setting: 390+564}
%\INSconfig{}{UVES}{DIC-1}{Standard setting: 346+564}
%\INSconfig{}{UVES}{DIC-1}{Standard setting: 390+580}
%\INSconfig{}{UVES}{DIC-1}{Non-std setting: provide central wavelength HERE}
%
%\INSconfig{}{UVES}{DIC-2}{Standard setting: 437+860}
%\INSconfig{}{UVES}{DIC-2}{Standard setting: 346+860}
%\INSconfig{}{UVES}{DIC-2}{Standard setting: 390+860}
%
%\INSconfig{}{UVES}{DIC-2}{Standard setting: 437+760}
%\INSconfig{}{UVES}{DIC-2}{Standard setting: 346+760}
%\INSconfig{}{UVES}{DIC-2}{Standard setting: 390+760}
%\INSconfig{}{UVES}{DIC-2}{Non-std setting: provide central wavelength HERE}
%
%\INSconfig{}{UVES}{Field Derotation}{yes}
%\INSconfig{}{UVES}{Image slicer-1}{yes}
%\INSconfig{}{UVES}{Image slicer-2}{yes}
%\INSconfig{}{UVES}{Image slicer-3}{yes}
%\INSconfig{}{UVES}{Iodine cell}{yes}
%\INSconfig{}{UVES}{Longslit Filters}{Provide list of filters HERE}
%
%\INSconfig{}{UVES}{RRM}{yes}
%
%-----------------------------------------------------------------------
%---- SPHERE at the VLT-UT3 (MELIPAL) -----------------------------------
%-----------------------------------------------------------------------
%
%
% Pupil or field tracking?
% Mode choices: IRDIS-CI, IRDIS-DBI, 
%               IRDIFS, IRDIFS-EXT, 
%               ZIMPOL-I
%               (Not relevant for IRDIS-DPI, IRDIS-LSS, ZIMPOL-P1 or ZIMPOL-P2)
%--------------------
% IRDIFS: 
% Coronagraph combination choices:
%   IRDIFS:     None, N-ALC-YJH-S, N-ALC-YJH-L, N-CLC-SW-L
%   IRDIFS-EXT: None, N-ALC-YJH-S, N-ALC-YJH-L, N-ALC-Ks
% Filter choices for IRDIS in IRDIFS mode
%   IRDIFS:     DB-H23, DB-ND23, DB-H34, BB-H
%   IRDIFS-EXT: DB-K12, BB-Ks
%---------------------
% IRDIS imaging (alone):
% Coronagraph combination choices for IRDIS imaging modes (see UM for details)
%   IRDIS-CI, IRDIS-DPI:  
%              None, N-ALC-Y, N-ALC-YJ-S, N-ALC-YJ-L, N-ALC-YJH-S, 
%                    N-ALC-YJH-L, N-ALC-Ks
%   IRDIS-DBI: None, N-ALC-Y, N-ALC-YJ-S, N-ALC-YJ-L, N-ALC-YJH-S, 
%                    N-ALC-YJH-L, N-ALC-Ks
% Filter choices:
%   IRDIS-CI, IRDIS-DPI: 
%              BB-Y, BB-J, BB-H, BB-Ks, NB-Hel, NB-CntJ, NB-CntH,
%              NB-CntK1, NB-BrG, NB-CntK2, NB-PaB, NB-FeII, NB-H2, NB-CO
%   IRDIS-DBI: DB-Y23, DB-J23, DB-H23, DB-NDH23,  DB-H34, DB-K12 
%---------------------
% IRDIS spectroscopy:
% Coronagraphic slit/grism combinations for IRDIS-LSS:
%   IRDIS-LSS: N-S-LR-WL, N-S-MR-WL
%---------------------
% ZIMPOL imaging: 
% Coronagraph choices:
%   ZIMPOL-I: None, V-CLC-M-WF, V-CLC-M-NF, V-CLC-L-WF, V-CLC-XL-WF
% Filter choices:
%   ZIMPOL-I: RI, R-PRIM, I-PRIM, V, V-S, V-L, N-R, 730-NB, N-I, I-L,
%             KI,  TiO-717, CH4-727, Cnt748, Cnt820, HeI, OI-630,
%             CntHa, B-Ha, N-Ha, Ha-NB
%--------------------
% ZIMPOL polarimetry:
% Coronagraph choices:
%    ZIMPOL-P1: None, V-CLC-S-WF, V-CLC-M-WF, V-CLC-L-WF, V-CLC-XL-WF, V-CLC-MT-WF
%    ZIMPOL-P2: None, V-CLC-S-WF, V-CLC-M-WF, V-CLC-L-WF, V-CLC-XL-WF, V-CLC-MT-WF
% Filter choices:
%    ZIMPOL-P1: RI, R-PRIM, I-PRIM, V, N-R, N-I, KI, TiO-717, 
%               CH4-727, Cnt748, Cnt820, CntHa, N-Ha, B-Ha     
%    ZIMPOL-P2: RI, R-PRIM, I-PRIM, V, N-R, N-I, KI, TiO-717, 
%               CH4-727, Cnt748, Cnt820, CntHa, N-Ha, B-Ha 
% Readout mode choice for ZIMPOL
%    ZIMPOL-P1: FastPol, SlowPol
%    ZIMPOL-P2: FastPol, SlowPol
%-------------------
%
% One entry per mode. Repeat the entry for each mode.
%
%\INSconfig{}{SPHERE}{Pupil}{mode}
%\INSconfig{}{SPHERE}{Field}{mode}
%
% One entry per combination. Repeat the entry for each combination.
%
%\INSconfig{}{SPHERE}{IRDIFS}{Coronagraph/filter combination for IRDIFS}
%\INSconfig{}{SPHERE}{IRDIFS-EXT}{Coronagraph/filter combination for IRDIFS-EXT}
%
%\INSconfig{}{SPHERE}{IRDIS-CI}{Coronagraph/filter combination for IRDIS-CI}
%\INSconfig{}{SPHERE}{IRDIS-DBI}{Coronagraph/filter combination for IRDIS-DBI}
%\INSconfig{}{SPHERE}{IRDIS-DPI}{Coronagraph/filter combination for IRDIS-DPI}
%\INSconfig{}{SPHERE}{IRDIS-LSS}{Coronagraphic slit/grism combination for IRDIS-LSS}
%
%\INSconfig{}{SPHERE}{ZIMPOL-I}{Coronagraph/filter combination for ZIMPOL-I}
%
%\INSconfig{}{SPHERE}{ZIMPOL-P1}{Coronagraph/filter/readout mode for ZIMPOL-P1}
%\INSconfig{}{SPHERE}{ZIMPOL-P2}{Coronagraph/filter/readout mode for ZIMPOL-P2}
%
%-----------------------------------------------------------------------
%---- VISIR at the VLT-UT3 (MELIPAL) -----------------------------------
%-----------------------------------------------------------------------
%
%
% List of offered filters for IMG:
%    M-BAND, J7.9, PAH1, J8.9, B8.7, ArIII, J9.8, SIV-1, B9.7, SIV, B10.7,
%    SIV-2, PAH2, B11.7, PAH2-2, J12.2, NeII-1, B12.4, NeII, NeII-2, Q1, Q2, Q3
%
%\INSconfig{}{VISIR}{IMG 45 mas/px}{Provide list of filters HERE}
%
% List of offered filters for CORONA AGPM:
%    10-5-4QP,11-3-4QP,12-3-AGP,12-4-AGP
%\INSconfig{}{VISIR}{CORONA 45 mas/px}{List of filters}
%
% List of filters offered for SAM:
%    10-5-SAM, 11-3-SAM
%
%\INSconfig{}{VISIR}{SAM 45 mas/px}{List of filters}
%
% Spectroscopy:
%
%\INSconfig{}{VISIR}{SPEC N-band LR}{-}
%\INSconfig{}{VISIR}{SPEC N-band HR Longslit}{Provide central wavelengt(s) (8.02,12.81) HERE}
%\INSconfig{}{VISIR}{SPEC Q-band HR Longslit}{Provide central wavelength(s) (17.03) HERE}
%\INSconfig{}{VISIR}{SPEC N-band HRCrossdispersed}{Provide central wavelength(s) (7.7-13.3)}
%\INSconfig{}{VISIR}{SPEC Q-band HRCrossdispersed}{Provide central wavelength(s) (16.0-24.0) HERE}
%
%-----------------------------------------------------------------------
%---- VIMOS at the VLT-UT3 (MELIPAL) -----------------------------------
%-----------------------------------------------------------------------
%
%\INSconfig{}{VIMOS}{PRE-IMG}{ESO filters: enter the list of filters}
%\INSconfig{}{VIMOS}{IMG}{ESO filters: enter the list of filters}
%\INSconfig{}{VIMOS}{IFU 0.67"/fibre}{LR-Red}
%\INSconfig{}{VIMOS}{IFU 0.67"/fibre}{LR-Blue}
%\INSconfig{}{VIMOS}{IFU 0.67"/fibre}{MR}
%\INSconfig{}{VIMOS}{IFU 0.67"/fibre}{HR-Red}
%\INSconfig{}{VIMOS}{IFU 0.67"/fibre}{HR-Orange}
%\INSconfig{}{VIMOS}{IFU 0.67"/fibre}{HR-Blue}
%
%\INSconfig{}{VIMOS}{IFU 0.33"/fibre}{LR-Red}
%\INSconfig{}{VIMOS}{IFU 0.33"/fibre}{LR-Blue}
%\INSconfig{}{VIMOS}{IFU 0.33"/fibre}{MR}
%\INSconfig{}{VIMOS}{IFU 0.33"/fibre}{HR-Red}
%\INSconfig{}{VIMOS}{IFU 0.33"/fibre}{HR-Orange}
%\INSconfig{}{VIMOS}{IFU 0.33"/fibre}{HR-Blue}
%
%\INSconfig{}{VIMOS}{MOS-grisms}{LR-Red}
%\INSconfig{}{VIMOS}{MOS-grisms}{LR-Blue}
%\INSconfig{}{VIMOS}{MOS-grisms}{MR}
%\INSconfig{}{VIMOS}{MOS-grisms}{HR-Red}
%\INSconfig{}{VIMOS}{MOS-grisms}{HR-Orange}
%\INSconfig{}{VIMOS}{MOS-grisms}{HR-Blue}
%
%\INSconfig{}{VIMOS}{MOS-slits-targets}{0.6" < slit width < 1.4", targets:stellar}
%\INSconfig{}{VIMOS}{MOS-slits-targets}{0.6" < slit width < 1.4", targets:extended}
%\INSconfig{}{VIMOS}{MOS-slits-targets}{slit width > 1.4", targets:stellar}
%\INSconfig{}{VIMOS}{MOS-slits-targets}{slit width > 1.4", targets:extended}
%\INSconfig{}{VIMOS}{MOS-masks}{Enter here number of mask sets (1 set = 4 quadrants)}
%
%
%%-----------------------------------------------------------------------
%---- HAWKI at the VLT-UT4 (YEPUN) -----------------------------------
%-----------------------------------------------------------------------
%
%\INSconfig{}{HAWKI}{PRE-IMG}{provide list of filters (Y,J,H,Ks,CH4,BrG,H2,NB1190,NB1060,NB2090) HERE}
%\INSconfig{}{HAWKI}{IMG}{provide list of filters (Y,J,H,Ks,CH4,BrG,H2,NB1190,NB1060,NB2090) HERE}
%\INSconfig{}{HAWKI}{BURST}{Provide list of filters  (Y,J,H,Ks,CH4,BrG,H2,NB1190,NB1060,NB2090) HERE}
%\INSconfig{}{HAWKI}{FASTJITT}{Provide list of filters  (Y,J,H,Ks,CH4,BrG,H2,NB1190,NB1060,NB2090) HERE}
%\INSconfig{}{HAWKI}{RRM}{yes}
%
%-----------------------------------------------------------------------
%---- SINFONI at the VLT-UT4 (YEPUN) -----------------------------------
%-----------------------------------------------------------------------
%

%\INSconfig{}{SINFONI}{PRE-IMG}{provide list of setting(s) (J,H,K,H+K)}
%
%\INSconfig{}{SINFONI}{IFS 250mas/pix no-AO}{provide list of setting(s) (J,H,K,H+K) HERE}
%\INSconfig{}{SINFONI}{IFS 100mas/pix no-AO}{provide list of setting(s) (J,H,K,H+K) HERE}
%
% If you plan to use a NGS, please specify the NGS name and magnitude (Rmag preferred,
% otherwise Vmag) in target list.
%\INSconfig{}{SINFONI}{IFS 250mas/pix NGS}{provide list of setting(s) (J,H,K,H+K) HERE}
%\INSconfig{}{SINFONI}{IFS 100mas/pix NGS}{provide list of setting(s) (J,H,K,H+K) HERE}
%\INSconfig{}{SINFONI}{IFS 25mas/pix NGS}{provide list of setting(s) (J,H,K,H+K) HERE}
%
% If you plan to use the LGS, please specify the TTS name and magnitude (Rmag preferred,
% otherwise Vmag) in target list.
%\INSconfig{}{SINFONI}{IFS 250mas/pix LGS}{provide list of setting(s) (J,H,K,H+K) HERE}
%\INSconfig{}{SINFONI}{IFS 100mas/pix LGS}{provide list of setting(s) (J,H,K,H+K) HERE}
%\INSconfig{}{SINFONI}{IFS 25mas/pix LGS}{provide list of setting(s) (J,H,K,H+K) HERE}
%
% If you plan to use the LGS without a TTS (seeing enhancer mode) then
% please leave the TTS name blank in the target list.
%\INSconfig{}{SINFONI}{IFS 250mas/pix LGS-noTTS}{provide list of setting(s) (J,H,K,H+K) HERE}
%\INSconfig{}{SINFONI}{IFS 100mas/pix LGS-noTTS}{provide list of setting(s) (J,H,K,H+K) HERE}
%\INSconfig{}{SINFONI}{IFS 25mas/pix LGS-noTTS}{provide list of setting(s) (J,H,K,H+K) HERE}
%
% Select if you have special calibrations
%\INSconfig{}{SINFONI}{Special Cal}{-}
%
% Select if you need pupil tracking mode
%\INSconfig{}{SINFONI}{Pupil Track}{-}
%
% Select for RRM
%\INSconfig{}{SINFONI}{RRM}{yes}
%
%-----------------------------------------------------------------------
%---- MUSE at the VLT-UT4 (YEPUN) -----------------------------------
%-----------------------------------------------------------------------
%
%\INSconfig{}{MUSE}{WFM-NOAO-N}{-}
%\INSconfig{}{MUSE}{WFM-NOAO-E}{-}
%
% Uncomment the following line for Rapid Response Mode observations
%\INSconfig{}{MUSE}{RRM}{yes}
%
%%%%%%%%%%%%%%%%%%%%%%%%%%%%%%%%%%%%%%%%%%%%%%%%%%%%%%%%%%%%%%%%%%%%%%%%
%-----------------------------------------------------------------------
%---- AMBER ------------------------------------------------------------
%-----------------------------------------------------------------------
%
%\INSconfig{}{AMBER}{LR-HK}{2.2}
%\INSconfig{}{AMBER}{LR-HK-F}{2.2}
%
%\INSconfig{}{AMBER}{MR-K}{2.1}
%\INSconfig{}{AMBER}{MR-K-F}{2.1}
%
%\INSconfig{}{AMBER}{MR-H}{1.65}   
%\INSconfig{}{AMBER}{MR-H-F}{1.65} 
%
%\INSconfig{}{AMBER}{MR-K}{2.3}
%\INSconfig{}{AMBER}{MR-K-F}{2.3}
%
%\INSconfig{}{AMBER}{HR-K}{Central wavelength selected from the list:
% 1.97929,2.01786,2.05643,2.09500,2.13357,2.17214,2.21071,2.24929,2.28786,2.32643,
% 2.36500,2.40357,2.44214,2.48071}
%\INSconfig{}{AMBER}{HR-K-F}{Central wavelength selected from the list:
% 1.97929,2.01786,2.05643,2.09500,2.13357,2.17214,2.21071,2.24929,2.28786,2.32643,
% 2.36500,2.40357,2.44214,2.48071}
% 
%where *-F means with FINITO
%

%-----------------------------------------------------------------------
%---- GRAVITY ----------------------------------------------------------
%-----------------------------------------------------------------------
%
%
%\INSconfig{}{GRAVITY}{Single-Field}{provide list of grating(s) (LR,MR,HR) HERE}
%\INSconfig{}{GRAVITY}{Dual-Field}{provide list of grating(s) (LR,MR,HR) HERE}
%
%
%-----------------------------------------------------------------------
%---- PIONIER ----------------------------------------------------------
%-----------------------------------------------------------------------
%
%
%\INSconfig{}{PIONIER}{GRISM}{1.65}
%\INSconfig{}{PIONIER}{FREE}{1.65}
%
%%%%%%%%%%%%%%%%%%%%%%%%%%%%%%%%%%%%%%%%%%%%%%%%%%%%%%%%%%%%%%%%%%%%%%%%
%
%-----------------------------------------------------------------------
%---- VIRCAM at VISTA --------------------------------------------------
%-----------------------------------------------------------------------
%
%\INSconfig{}{VIRCAM}{IMG}{provide list of filters here}
%
%-----------------------------------------------------------------------
%---- OMEGACAM at VST --------------------------------------------------
% This instrument is only available for GTO, Chilean and filler programmes.
%-----------------------------------------------------------------------
%
%\INSconfig{}{OMEGACAM}{IMG}{provide list of filters here}
%
%%%%%%%%%%%%%%%%%%%%%%%%%%%%%%%%%%%%%%%%%%%%%%%%%%%%%%%%%%%%%%%%%%%%%%%%
% La Silla
%-----------------------------------------------------------------------
%---- EFOSC2 (or SOFOSC) at the NTT ------------------------------------
%-----------------------------------------------------------------------
%
%\INSconfig{}{EFOSC2}{PRE-IMG}{EFOSC2 filters: provide list here}
%\INSconfig{}{EFOSC2}{Imaging-filters}{EFOSC2 filters:  provide list here}
%\INSconfig{}{EFOSC2}{Imaging-filters}{ESO non EFOSC filters: provide ESOfilt No}
%\INSconfig{}{EFOSC2}{Imaging-filters}{User's own filters (to be described in text)}
%\INSconfig{}{EFOSC2}{Spectro-long-slit}{Grism\#1:320-1090}
%\INSconfig{}{EFOSC2}{Spectro-long-slit}{Grism\#2:510-1100}
%\INSconfig{}{EFOSC2}{Spectro-long-slit}{Grism\#3:305-610}
%\INSconfig{}{EFOSC2}{Spectro-long-slit}{Grism\#4:409-752}
%\INSconfig{}{EFOSC2}{Spectro-long-slit}{Grism\#5:520-935}
%\INSconfig{}{EFOSC2}{Spectro-long-slit}{Grism\#6:386-807}
%\INSconfig{}{EFOSC2}{Spectro-long-slit}{Grism\#7:327-524}
%\INSconfig{}{EFOSC2}{Spectro-long-slit}{Grism\#8:432-636}
%\INSconfig{}{EFOSC2}{Spectro-long-slit}{Grism\#11:338-752}
%\INSconfig{}{EFOSC2}{Spectro-long-slit}{Grism\#13:369-932}
%\INSconfig{}{EFOSC2}{Spectro-long-slit}{Grism\#14:310-509}
%\INSconfig{}{EFOSC2}{Spectro-long-slit}{Grism\#16:602-1032}
%\INSconfig{}{EFOSC2}{Spectro-long-slit}{Grism\#17:689-876}
%\INSconfig{}{EFOSC2}{Spectro-long-slit}{Grism\#18:470-677}
%\INSconfig{}{EFOSC2}{Spectro-long-slit}{Grism\#19:440-510}
%\INSconfig{}{EFOSC2}{Spectro-long-slit}{Grism\#20:605:715}
%\INSconfig{}{EFOSC2}{Spectro-long-slit}{Aperture: 0.5'', ... ,10.0''}
%
%\INSconfig{}{EFOSC2}{Spectro-long-slit}{Aperture: Shiftable}
%\INSconfig{}{EFOSC2}{Spectro-MOS}{PunchHead=0.95''}
%\INSconfig{}{EFOSC2}{Spectro-MOS}{PunchHead=1.12''}
%\INSconfig{}{EFOSC2}{Spectro-MOS}{PunchHead=1.45''}
%\INSconfig{}{EFOSC2}{Polarimetry}{$\lambda / 2$ retarder plate}
%\INSconfig{}{EFOSC2}{Polarimetry}{$\lambda / 4$ retarder plate}
%\INSconfig{}{EFOSC2}{Coronograph}{yes}
%
%
%-----------------------------------------------------------------------
%---- SOFI (or SOFOSC) at the NTT --------------------------------------------------
%-----------------------------------------------------------------------
%
%\INSconfig{}{SOFI}{PRE-IMG-LargeField}{Provide list of filters HERE}
%\INSconfig{}{SOFI}{Imaging-LargeField}{Provide list of filters HERE}
%\INSconfig{}{SOFI}{Burst}{Provide list of filters HERE}
%\INSconfig{}{SOFI}{FastPhot}{Provide list of filters HERE}
%\INSconfig{}{SOFI}{Polarimetry}{Provide list of filters HERE}
%\INSconfig{}{SOFI}{Spectroscopy-long-slit}{Blue Grism, Provide list of slits HERE}
%\INSconfig{}{SOFI}{Spectroscopy-long-slit}{Red Grism, Provide list of slits HERE}
%\INSconfig{}{SOFI}{Spectroscopy-high-res}{H, Provide list of slits HERE}
%\INSconfig{}{SOFI}{Spectroscopy-high-res}{K, Provide list of slits HERE}
%
%
%-----------------------------------------------------------------------
%---- HARPS at the 3.6 -------------------------------------------------
%-----------------------------------------------------------------------
%
%\INSconfig{}{HARPS}{spectro-Thosimult}{HARPS}
%\INSconfig{}{HARPS}{WAVE}{HARPS}
%\INSconfig{}{HARPS}{spectro-ObjA(B)}{HARPS}
%\INSconfig{}{HARPS}{spectro-ObjA(B)}{EGGS}
%\INSconfig{}{HARPS}{spectro-polarimetry}{linear}
%\INSconfig{}{HARPS}{spectro-polarimetry}{circular}
%
%
%%%%%%%%%%%%%%%%%%%%%%%%%%%%%%%%%%%%%%%%%%%%%%%%%%%%%%%%%%%%%%%%%%%%%%%%
% Chajnantor
%-----------------------------------------------------------------------
%---- SHFI at APEX ----------------------------------------------
%-----------------------------------------------------------------------
%
%\INSconfig{}{SHFI}{APEX-1}{Please enter Central Frequency 211 to 275 GHz}
%\INSconfig{}{SHFI}{APEX-2}{Please enter Central Frequency 275 to 370 GHz}
%\INSconfig{}{SHFI}{APEX-3}{Please enter Central Frequency 385 to 500 GHz} 
%
%-----------------------------------------------------------------------
%---- LABOCA at APEX ----------------------------------------------
%-----------------------------------------------------------------------
%
%\INSconfig{}{LABOCA}{IMG}{-}
%
%-----------------------------------------------------------------------
%---- Artemis at APEX ----------------------------------------------
%-----------------------------------------------------------------------
%
%\INSconfig{}{ARTEMIS}{IMG}{350 um}
%
%-----------------------------------------------------------------------
%---- FLASH at APEX ----------------------------------------------
%-----------------------------------------------------------------------
%
%\INSconfig{}{FLASH}{-}{Please enter Central Frequency 272 to 377 GHz and 385 to 495 GHz}
%
%-----------------------------------------------------------------------
%-----------------------------------------------------------------------
%---- SEPIA at APEX ----------------------------------------------
%-----------------------------------------------------------------------
%
%\INSconfig{}{SEPIA}{Band-5}{Please enter Central Frequency 159 to 211 GHz}
%\INSconfig{}{SEPIA}{Band-9}{Please enter Central Frequency 602 to 720 GHz}
%



%%%%%%%%%%%%%%%%%%%%%%%%%%%%%%%%%%%%%%%%%%%%%%%%%%%%%%%%%%%%%%%%%%%%%%%%
%%%%% Interferometry PAGE %%%%%%%%%%%%%%%%%%%%%%%%%%%%%%%%%%%%%%%%%%%%%%
%%%%%%%%%%%%%%%%%%%%%%%%%%%%%%%%%%%%%%%%%%%%%%%%%%%%%%%%%%%%%%%%%%%%%%%%
%
% The \VLTITarget macro is only needed when requesting
% Interferometry, in which case it is MANDATORY to uncomment it and
% fill in the information. It takes the following parameters:
%
% 1st argument: run ID
% Valid values: run IDs specified in BOX 3
%
% 2nd argument: target name
% This parameter is NOT checked at the pdfLaTeX compilation.
%
% 3rd argument: visual magnitude
% Values with up to decimal places are allowed here.
% This parameter is NOT checked at the pdfLaTeX compilation.
%
% 4th argument: magnitude at wavelength of observation
% Values with up to decimal places are allowed here.
% This parameter is NOT checked at the pdfLaTeX compilation.
%
% 5th argument: wavelength of observation (in microns)
% Values with up to decimal places are allowed here.
% This parameter is NOT checked at the pdfLaTeX compilation.
%
% 6th argument: size at wavelength of observation (in mas)
% This parameter is NOT checked at the pdfLaTeX compilation.
%
% 7th argument: baseline
% UT observations are scheduled in terms of 3-telescope
% baselines for AMBER and 4-telescope baselines for PIONIER
% and GRAVITY.
% For UT observations please specify one of the available
% AMBER triplets or a PIONIER/GRAVITY quadruplet.
%
% AT observations are scheduled in terms of 4-telescope
% configurations (quadruplets) for any instrument. For AMBER,
% the time can be split among the different available triplets
% at Phase 2.
% For AT observations with any instrument, please specify 
% one of the 3 available AT quadruplets at this stage.
%
%
% 8th parameter: Range of visibilities for the specified configuration.
% Please specify the maximum and minimum visibility values
% corresponding to the chosen configuration at hour angle 0
% separated by "/".
% This parameter is NOT checked at the pdfLaTeX compilation. 
%
% 9th parameter: correlated magnitude
% (for the visibility values specified in the 8th parameter)
% This parameter is NOT checked at the pdfLaTeX compilation.
%
%
% 10th parameter: time on target in hours
% Values with up to decimal places are allowed here.
% This parameter is NOT checked at the pdfLaTeX compilation.
%
% The available baselines for Period 99 are shown below.
% For AT observations with AMBER, the time can be split 
% at Phase 2 among the different offered triplets of the 
% chosen quadruplet. 
%
%
% 
% AMBER
% A0-B2-C1-D0
% A0-G1-J2-J3
% D0-G2-J3-K0
% UT1-UT2-UT3
% UT1-UT2-UT4
% UT1-UT3-UT4
% UT2-UT3-UT4
% 
% GRAVITY
% A0-B2-C1-D0
% A0-G1-J2-J3
% D0-G2-J3-K0
% UT1-UT2-UT3-UT4
% 
% PIONIER
% A0-B2-C1-D0
% A0-G1-J2-J3
% D0-G2-J3-K0
% UT1-UT2-UT3-UT4
% 

%\VLTITarget{E}{Alpha Ori}{-1.4}{-1.4}{10.6}{6}{UT1-UT2-UT3}{0.60/0.10}{-0.2/4.0}{2} 
%\VLTITarget{F}{Alpha Ori}{-1.4}{-1.4}{10.6}{6}{D0-K0-G2-J3}{0.80/0.40}{-0.9/-0.2}{1} 

% You can specify here a note applying to all or some of your VLTI
% targets.  You should take advantage of this note to indicate
% suitable alternative baselines for your observations.
% This macro is NOT checked at the pdfLaTeX compilation.

%\VLTITargetNotes{Note about the VLTI targets, e.g., Run E can also be carried out using UT1-UT3-UT4.}

%%%%%%%%%%%%%%%%%%%%%%%%%%%%%%%%%%%%%%%%%%%%%%%%%%%%%%%%%%%%%%%%%%%%%%%%
%%%%% ToO PAGE %%%%%%%%%%%%%%%%%%%%%%%%%%%%%%%%%%%%%%%%%%%%%%%%%%%%%%%%%
%%%%%%%%%%%%%%%%%%%%%%%%%%%%%%%%%%%%%%%%%%%%%%%%%%%%%%%%%%%%%%%%%%%%%%%%
%
% The \ToOrun macro is needed only when requesting Target of
% Opportunity (ToO) observations, in which case it is MANDATORY to
% uncomment it and fill in the information. It takes the following
% parameters: 
%
% 1st argument: run ID
% Valid values: run IDs specified in BOX 3
%
% 2nd argument: nature of observation
% Valid values: RRM, ToO-hard, ToO-soft
%
% 3rd argument: number of targets per run
% This parameter is NOT checked at the pdfLaTeX compilation.
%
% 4th argument: number of triggers per targets
% (for RRM and ToO observations only)
% This parameter is NOT checked at the pdfLaTeX compilation.

%\TOORun{A}{RRM}{2}{3}
%\TOORun{B}{ToO-hard}{3}{1}

% You have the opportunity to add notes to the ToO runs by using
% the \TOONotes macro.
% This macro is NOT checked at the pdfLaTeX compilation.

%\TOONotes{Use this macro to add a note to the ToO page.}


%%%%%%%%%%%%%%%%%%%%%%%%%%%%%%%%%%%%%%%%%%%%%%%%%%%%%%%%%%%%%%%%%%%%%%%%
%%%%% VISITOR SPECIAL INSTRUMENT PAGE %%%%%%%%%%%%%%%%%%%%%%%%%%%%%%%%%%
%%%%%%%%%%%%%%%%%%%%%%%%%%%%%%%%%%%%%%%%%%%%%%%%%%%%%%%%%%%%%%%%%%%%%%%%
%
% The following commands are only needed when bringing a Visitor
% Special Instrument, in which case it is MANDATORY to uncomment them
% and provide all the required information.
%
%\Desc{}   %Description of the instrument and its operation
%\Comm{}   %On which telescope(s) has instrument been commissioned/used
%\WV{}     %Total weight and value of equipment to be shipped
%\Wfocus{} %Weight at the focus (including ancillary equipment)
%\Interf{} %Compatibility of attachment interface with required focus
%\Focal{}  %Back focal distance value
%\Acqu{}   %Acquisition, focusing, and guiding procedure
%\Softw{}  %Compatibility with ESO software standards (data handling)
%\Suppl{}  %Estimate of services expected from ESO (in person days)

%%%%%%%%%%%%%%%%%%%%%%%%%%%%%%%%%%%%%%%%%%%%%%%%%%%%%%%%%%%%%%%%%%%%%%%%
%%%%% THE END %%%%%%%%%%%%%%%%%%%%%%%%%%%%%%%%%%%%%%%%%%%%%%%%%%%%%%%%%%
%%%%%%%%%%%%%%%%%%%%%%%%%%%%%%%%%%%%%%%%%%%%%%%%%%%%%%%%%%%%%%%%%%%%%%%%
\MakeProposal
\end{document}


